% ============================================================================
% Paper III: Fermion Mass Hierarchy from K_8 on a Genus-2 Surface
% ============================================================================
\documentclass[12pt,a4paper]{article}

\usepackage{amsmath,amssymb,amsthm}
\usepackage{geometry}
\usepackage{graphicx}
\usepackage{booktabs}
\usepackage{hyperref}
\usepackage{cleveref}
\usepackage{mathtools}
\usepackage{xcolor}
\usepackage{enumitem}
\usepackage{float}

\geometry{margin=1in}

% Theorem environments
\newtheorem{theorem}{Theorem}[section]
\newtheorem{proposition}[theorem]{Proposition}
\newtheorem{lemma}[theorem]{Lemma}
\newtheorem{corollary}[theorem]{Corollary}
\newtheorem{conjecture}[theorem]{Conjecture}
\theoremstyle{definition}
\newtheorem{definition}[theorem]{Definition}
\theoremstyle{remark}
\newtheorem{remark}[theorem]{Remark}

% Shortcuts
\newcommand{\R}{\mathbb{R}}
\newcommand{\C}{\mathbb{C}}
\newcommand{\N}{\mathbb{N}}
\newcommand{\Z}{\mathbb{Z}}
\newcommand{\Tr}{\operatorname{Tr}}
\newcommand{\Pf}{\operatorname{Pf}}
\newcommand{\so}{\mathfrak{so}}
\newcommand{\su}{\mathfrak{su}}
\newcommand{\diag}{\operatorname{diag}}
\newcommand{\rank}{\operatorname{rank}}

% BibTeX
\usepackage[numbers,sort&compress]{natbib}

\title{Fermion Mass Hierarchy from $K_8$ on a Genus-2 Surface}
\author{Brian Porter}
\date{February 2026}

\begin{document}

\maketitle

\begin{abstract}
We extend the spectral action on complete graphs to $K_8$, whose minimal orientable embedding requires a genus-2 surface~$\Sigma_2$.
The 105 perfect matchings of $K_8$ carry a direction structure inherited from the Heawood embedding of $K_7$ on the torus: three torus direction classes from the $\Z_7$-symmetric Heffter map, plus a fourth class of seven handle edges connecting the genus-raising vertex.
The resulting $105 \times 105$ Gram matrix has a six-dimensional physical vacuum eigenspace at $\lambda_{\mathrm{vac}} = 1.9595$.
Under the $\Z_7$ automorphism of the Heawood map, this eigenspace decomposes as
$V_{\mathrm{vac}} = V_{\rho_1} \oplus V_{\rho_2} \oplus V_{\rho_3}$,
three conjugate pairs of irreducible representations with zero trivial component.

The physical $K_6$ vacuum---the eigenvector giving $m_H = 125$~GeV in the companion paper---embeds in $K_8$ via hub matchings containing the doublet edge $(0,1)$.
Its projection onto $V_{\mathrm{vac}}$ is \emph{purely} $\rho_1$ (100\%), producing a rank-3 Yukawa matrix with eigenvalue hierarchy $415 : 135 : 1$ at zero free parameters.
No $K_6$ eigenvector projects onto~$\rho_2$, establishing a selection rule that decouples one sector from the Higgs mechanism entirely.
This structural inaccessibility provides a geometric origin for the qualitative difference between charged fermion and neutrino mass generation.

The $3 \times 3$ Yukawa matrix from the $K_6$-determined vacuum direction has three nonzero eigenvalues for five of seven possible doublet positions within $K_7$, with hierarchies ranging from $25:2:1$ to $544:74:1$.
The middle eigenvalue ratio $135:1$ sits within 1.5\% of the Standard Model charm-to-up mass-squared ratio $m_c^2/m_u^2 \approx 137$.
Within the full six-dimensional vacuum eigenspace, a random scan finds the down-type quark ratio $790:20:1$ reproduced as $768:20:1$ (3\% accuracy on $m_b^2/m_d^2$, 1\% on $m_s^2/m_d^2$).

Five parameter-free structural predictions emerge:
(1)~three massive fermion generations is a generic outcome (99.99\% of vacuum directions);
(2)~the hierarchy scale is $10^2$--$10^3$, matching the Standard Model;
(3)~one of three $\Z_7$ sectors is Higgs-decoupled;
(4)~the Higgs-coupled sector accessible to the $K_6$ ground state has rank~3;
(5)~the sector accessible only to excited $K_6$ states has rank~2.
\end{abstract}

\tableofcontents
\newpage


% ============================================================================
% SECTION 1: INTRODUCTION
% ============================================================================
\section{Introduction}\label{sec:intro}

The origin of fermion mass hierarchies---spanning five orders of magnitude from the electron to the top quark---is among the deepest open problems in particle physics.
The Standard Model accommodates these masses through Yukawa couplings to the Higgs field, but provides no explanation for their values or the striking pattern of three generations with exponentially separated masses.

In the Chamseddine--Connes noncommutative geometry (NCG) framework~\cite{Chamseddine2007,Connes1996,CCM}, the Standard Model Lagrangian emerges from a spectral action principle applied to a product geometry $M^4 \times F$, where $F$ is a finite noncommutative space encoding the internal degrees of freedom.
The spectral action determines gauge couplings, the Higgs potential, and in principle the Yukawa couplings---but the Yukawa sector of the finite space $F$ has traditionally been treated as input rather than output.

In companion papers~\cite{PaperI,PaperII}, we developed a framework in which the finite geometry is constructed from complete graphs $K_n$.
Paper~I showed that Lorentzian signature $(1,3)$ emerges from the Pfaffian structure of $K_4$.
The spectral action on $K_6$---the unique non-degenerate complete graph embeddable on a torus---yields a zero-parameter Higgs mass prediction $m_H \approx 125$~GeV through the ratio of Seeley--DeWitt coefficients $a_4/a_2^2 = 0.3722$, constrained by the combinatorial rigidity of matching overlaps.
However, $K_6$ explicitly \emph{fails} at fermion masses: its Yukawa matrix has rank~2 with a hierarchy of only $64:1$, nine orders of magnitude short of the Standard Model's $\sim 10^9:1$.

This paper addresses the fermion mass problem by advancing to $K_8$.
The complete graph on eight vertices has 28 edges and 105 perfect matchings.
Its minimal orientable embedding requires a genus-2 surface $\Sigma_2$ (double torus), as dictated by the Ringel--Youngs formula $\gamma(K_8) = \lceil (8-3)(8-4)/12 \rceil = 2$.
The passage from genus~1 to genus~2 provides the additional topological structure needed: a fourth direction class from the second handle, promoting the Yukawa from rank~2 to rank~3.

The central results of this paper are:
\begin{enumerate}[label=(\roman*)]
  \item The 28 edges of $K_8$ decompose into four classes of seven, using the $\Z_7$-symmetric Heawood classification of $K_7$ edges on the torus plus the seven handle edges connecting the genus-raising eighth vertex.
  \item The $105 \times 105$ Gram matrix has a six-dimensional vacuum eigenspace that decomposes as $\rho_1 \oplus \rho_2 \oplus \rho_3$ under $\Z_7$, with the trivial representation absent.
  \item The $K_6$ Higgs vacuum projects purely into $\rho_1$, yielding a zero-parameter Yukawa hierarchy $415:135:1$.
  \item No $K_6$ eigenvector reaches $\rho_2$---a proven selection rule that geometrically decouples one fermion sector from the Higgs.
  \item The Yukawa hierarchy from the full vacuum eigenspace spans all three Standard Model fermion sectors.
\end{enumerate}


% ============================================================================
% SECTION 2: K8 ON GENUS 2
% ============================================================================
\section{\texorpdfstring{$K_8$}{K8} on the genus-2 surface}\label{sec:k8genus2}

\subsection{Topological prerequisites}

The orientable genus of $K_n$ is given by the Ringel--Youngs formula~\cite{Ringel1968}:
\begin{equation}\label{eq:genus}
  \gamma(K_n) = \left\lceil \frac{(n-3)(n-4)}{12} \right\rceil, \qquad n \geq 3.
\end{equation}
For $n = 8$: $(8-3)(8-4)/12 = 20/12 \approx 1.667$, giving $\gamma(K_8) = 2$.
Thus $K_8$ is the smallest complete graph requiring a double torus for crossing-free embedding.
For comparison, $K_7$ fits on the torus ($\gamma = 1$) via the Heawood map, while $K_9$ requires genus~3.

A genus-2 embedding of $K_8$ satisfies the Euler relation:
\begin{equation}
  v - e + f = 2 - 2g = -2, \qquad v = 8,\; e = 28 \implies f = 18,
\end{equation}
yielding 18 faces (generically 16 triangles and 2 quadrilaterals).
On the torus ($g = 1$), $K_8$ would require $f = 20$ faces but only 28 edges to bound them; the inequality $2e = 56 < 3f = 60$ makes this impossible, confirming that genus~2 is necessary.

\subsection{The Heawood direction structure}\label{sec:heawood}

$K_7$ admits a unique (up to isomorphism) triangular embedding on the torus: the Heawood map.
Its $\Z_7$ rotational symmetry---the cyclic permutation $i \mapsto i + 1 \pmod{7}$---classifies the 21 edges into three direction classes of seven:
\begin{equation}\label{eq:heawood_dirs}
  d_0: |i - j| \equiv \pm 1 \pmod{7}, \qquad
  d_1: |i - j| \equiv \pm 2 \pmod{7}, \qquad
  d_2: |i - j| \equiv \pm 3 \pmod{7}.
\end{equation}
This is the direct analogue of the three lattice directions on $K_6$'s hexagonal torus that produced the Higgs mass.

Adding the eighth vertex (labeled~7) to form $K_8$ introduces seven new edges $\{(i, 7) : i = 0, \ldots, 6\}$.
These edges connect to the second handle of the genus-2 surface and form a single $\Z_7$ orbit.
We designate them as the fourth direction class:
\begin{equation}\label{eq:handle_dir}
  d_3: \text{handle edges } (i, 7), \quad i = 0, \ldots, 6.
\end{equation}

\begin{proposition}[Direction partition]\label{prop:partition}
The 28 edges of $K_8$ partition into four classes of seven edges each:
\[
  |d_0| = |d_1| = |d_2| = |d_3| = 7, \qquad
  \sum_{k=0}^{3} |d_k| = 28.
\]
This is the unique $\Z_7$-symmetric partition extending the Heawood classification of $K_7$.
\end{proposition}


\subsection{Matching structure}\label{sec:matchings}

$K_8$ has $|M(K_8)| = (8-1)!! = 105$ perfect matchings.
Each matching consists of four disjoint edges covering all eight vertices.
Since vertex~7 must be paired in every matching, each matching contains exactly one handle edge from $d_3$.

\begin{lemma}[Handle universality]\label{lem:handle}
Every perfect matching of $K_8$ contains exactly one edge from direction class $d_3$ and three edges from the torus classes $d_0, d_1, d_2$.
\end{lemma}

\begin{proof}
Vertex~7 appears in exactly one edge per matching.  The only edges incident to vertex~7 are the handle edges $(i, 7) \in d_3$.
\end{proof}

This universality has a profound consequence: the handle direction $d_3$ does not distinguish between matchings.
Any phase assigned to $d_3$ multiplies all matchings by the same factor, rendering it gauge-redundant in the Yukawa coupling (which depends on $Y^\dagger Y$).


% ============================================================================
% SECTION 3: THE GRAM MATRIX
% ============================================================================
\section{The Gram matrix}\label{sec:gram}

\subsection{Construction}

Following the $K_6$ framework, we construct the Dirac operator from perfect matchings.
Each matching $M_i$ is encoded as an $8 \times 8$ adjacency matrix with $(M_i)_{ab} = 1$ if vertices $a, b$ are paired, and~0 otherwise.
The Dirac operator at Brillouin zone momentum $k \in \R^4$ is:
\begin{equation}\label{eq:dirac}
  D(k) = \sum_{i=1}^{105} t_i \, \zeta_i \, e^{ik \cdot d_i} \, M_i,
\end{equation}
where $t_i \in \R$ are moduli ($\|t\|^2 = 1$), $\zeta_i$ is a $\Z_3$ phase determined by the torus direction of $M_i$, and $d_i \in \R^4$ is the lattice direction vector.

The lattice vectors for the four direction classes are:
\begin{equation}
  \vec{d}_0 = (1,0,0,0), \quad \vec{d}_1 = (0,1,0,0), \quad
  \vec{d}_2 = (-1,-1,0,0), \quad \vec{d}_3 = (0,0,1,0),
\end{equation}
where the first two components span the torus (first handle) and the third component addresses the second handle.
The net direction of matching $M_i$ is $D_i = \sum_{e \in M_i} \vec{d}_{c(e)}$, where $c(e)$ is the direction class of edge~$e$.

\subsection{Phase assignment}

Each matching receives a $\Z_3$ phase from its torus edge content.
The phase of edge $(a, b) \in d_k$ is $\omega^k$ where $\omega = e^{2\pi i/3}$, and the total phase of matching $M_i$ is:
\begin{equation}\label{eq:phase}
  \zeta_i = \prod_{e \in M_i} \omega^{c(e)}.
\end{equation}
By Lemma~\ref{lem:handle}, the handle direction $d_3$ contributes the same factor $\omega^{3} = 1$ (or $\omega$, etc.) to every matching, confirming gauge redundancy.

\subsection{Gram matrix entries}\label{sec:gram_entries}

The Seeley--DeWitt coefficient $a_2 = \langle \Tr(D^\dagger D) \rangle_{\mathrm{BZ}}$ is a quadratic form $a_2 = t^T G t$, with Gram matrix:
\begin{equation}\label{eq:gram}
  G_{ij} = \Tr(M_i M_j) \cdot \operatorname{Re}(\zeta_j^* \zeta_i) \cdot \delta(D_i, D_j),
\end{equation}
where $\delta(D_i, D_j)$ enforces Brillouin zone momentum conservation: the integral $\langle e^{ik \cdot (D_i - D_j)} \rangle_{\mathrm{BZ}}$ vanishes unless $D_i = D_j$.

The matching overlap $\Tr(M_i M_j)$ takes values in $\{0, 2, 4, 8\}$:
\begin{center}
\begin{tabular}{lcl}
  \toprule
  \textbf{Overlap} & \textbf{Count} & \textbf{Meaning} \\
  \midrule
  8 ($i = j$) & 105 & Self-overlap \\
  4 & 630 & Share 2 edges \\
  2 & 1680 & Share 1 edge \\
  0 & 3150 & Disjoint \\
  \bottomrule
\end{tabular}
\end{center}
This $3150:1680:630:105$ distribution is a graph invariant of $K_8$.


\subsection{Eigenspectrum}

The $105 \times 105$ Gram matrix $G$ has:
\begin{itemize}
  \item 5 zero eigenvalues (null space from momentum conservation),
  \item 100 positive eigenvalues, organized in multiplets of dimension~6 (from $\Z_7$ irrep structure) and sporadic dimensions 3--4.
\end{itemize}

The physical vacuum is the smallest positive eigenvalue:
\begin{equation}\label{eq:vacuum}
  \lambda_{\mathrm{vac}} = 1.9595, \qquad \text{multiplicity } 6.
\end{equation}

Nearly all eigenspaces have dimension~6, reflecting the decomposition of each into three conjugate pairs of $\Z_7$ representations.
The full spectrum spans from $\lambda_{\mathrm{vac}} = 1.9595$ to $\lambda_{\max} = 24.0$.


% ============================================================================
% SECTION 4: Z7 SYMMETRY AND VACUUM DECOMPOSITION
% ============================================================================
\section{\texorpdfstring{$\Z_7$}{Z7} symmetry and the vacuum eigenspace}\label{sec:z7}

\subsection{The \texorpdfstring{$\Z_7$}{Z7} action on matchings}

The Heawood map's cyclic symmetry acts on $K_8$ as $\sigma: i \mapsto i + 1 \pmod{7}$ with $7 \mapsto 7$ fixed.
This induces a permutation on the 105 matchings via:
\begin{equation}
  \sigma(M) = \{(\sigma(a), \sigma(b)) : (a, b) \in M\}.
\end{equation}
The resulting $105 \times 105$ permutation matrix $P$ satisfies $P^7 = I$ and commutes with the Gram matrix: $[P, G] = 0$.

The 105 matchings decompose into 15 orbits of size~7 under $\Z_7$, confirming:
\begin{equation}
  105 = 7 \times 15 = |\Z_7| \times |M(K_6)|.
\end{equation}
Each orbit of seven matchings contains a ``representative'' that, together with the hub edge $(0,1)$, corresponds to a $K_6$ matching on the remaining vertices $\{2,3,4,5,6,7\}$.

\subsection{Irreducible decomposition of the vacuum}

$\Z_7$ has one trivial and three conjugate pairs of irreducible representations $\rho_k$ ($k = 1, 2, 3$), each two-dimensional over $\R$.
The character is $\chi_k(\sigma^j) = \omega_7^{kj} + \omega_7^{-kj} = 2\cos(2\pi k j/7)$, where $\omega_7 = e^{2\pi i/7}$.

Projecting the six-dimensional vacuum eigenspace onto $\Z_7$ irreps using the projectors
\begin{equation}
  \Pi_k = \frac{1}{7} \sum_{j=0}^{6} \omega_7^{-kj} P^j
\end{equation}
yields:

\begin{theorem}[Vacuum decomposition]\label{thm:decomp}
The vacuum eigenspace decomposes as
\begin{equation}\label{eq:decomp}
  V_{\mathrm{vac}} = V_{\rho_1} \oplus V_{\rho_2} \oplus V_{\rho_3}, \qquad
  \dim V_{\rho_k} = 2 \quad (k = 1,2,3),
\end{equation}
with $V_{\rho_0} = \{0\}$ (the trivial representation is absent).
\end{theorem}

The absence of the trivial representation means the vacuum \emph{must} break $\Z_7$: no $\Z_7$-invariant vacuum state exists.
This symmetry breaking is not imposed but forced by the eigenspace structure.


% ============================================================================
% SECTION 5: THE YUKAWA COUPLING
% ============================================================================
\section{The Yukawa coupling}\label{sec:yukawa}

\subsection{Generation structure}

Following the $K_6$ framework, we designate two vertices as the Higgs doublet and partition the remaining six into three generation pairs.
The canonical assignment is:
\begin{equation}
  \text{Doublet: } \{0, 1\}, \qquad
  \text{Generations: } (2,3),\; (4,5),\; (6,7).
\end{equation}
Generation~2 contains vertex~7 (the handle vertex), making it topologically distinguished.

\subsection{Hub and non-hub matchings}

The 105 matchings divide into:
\begin{itemize}
  \item \textbf{Hub matchings} (15): contain edge $(0,1)$.  These embed the $K_6$ matchings on $\{2,3,4,5,6,7\}$.
  \item \textbf{Non-hub matchings} (90): pair doublet vertices $0, 1$ with generation vertices.  These carry the Yukawa coupling.
\end{itemize}

Each non-hub matching has the form $\{(0, a), (1, b)\} \cup M_{\mathrm{rest}}$, where $a$ and $b$ belong to different generation pairs (or the same pair for diagonal Yukawa entries).
The generation pair containing~$a$ and the pair containing~$b$ determine the Yukawa matrix element:
\begin{equation}\label{eq:yukawa_def}
  Y_{\alpha\beta} = \sum_{\substack{M \in \text{non-hub} \\ M \text{ couples } \alpha \leftrightarrow \beta}}
  v_M \, \zeta_M,
\end{equation}
where $v_M$ is the vacuum eigenvector component and $\zeta_M$ is the $\Z_3$ phase.

The counting of non-hub matchings per Yukawa element is:
\begin{center}
\begin{tabular}{lcc}
  \toprule
  \textbf{Element} & \textbf{Type} & \textbf{Count} \\
  \midrule
  $Y_{00}, Y_{11}, Y_{22}$ & Diagonal & 6 each \\
  $Y_{01}, Y_{10}, \ldots$ & Off-diagonal & 12 each \\
  \bottomrule
\end{tabular}
\end{center}
Total: $3 \times 6 + 6 \times 12 = 90$, consistent with $105 - 15 = 90$ non-hub matchings.


% ============================================================================
% SECTION 6: K6 VACUUM PROJECTION
% ============================================================================
\section{\texorpdfstring{$K_6$}{K6} vacuum determines the \texorpdfstring{$K_8$}{K8} Yukawa}\label{sec:k6proj}

\subsection{Embedding the \texorpdfstring{$K_6$}{K6} vacuum}

The 15 hub matchings of $K_8$ are in bijection with the 15 perfect matchings of $K_6$ on $\{2,3,4,5,6,7\}$: each $K_6$ matching~$m$ extends to $m \cup \{(0,1)\}$.
This embedding maps the $K_6$ Gram matrix eigenvalue problem into the hub sector of $K_8$.

The physical $K_6$ vacuum---the ground eigenvector at $\lambda_{6,\min} = 2.764$, whose spectral ratio $a_4/a_2^2 = 0.3722$ yields $m_H = 125$~GeV---is a 15-dimensional vector.
Embedded in $K_8$'s 105-dimensional matching space (with zeros for all non-hub components), it can be projected onto the $K_8$ vacuum eigenspace.

\subsection{The projection}

\begin{theorem}[$K_6$ selection rule]\label{thm:selection}
The physical $K_6$ vacuum, embedded via hub matchings, projects onto the $K_8$ vacuum eigenspace with:
\begin{equation}
  \|\Pi_{\mathrm{vac}} \, v_{K_6}\| = 0.2355, \qquad
  \frac{\|\Pi_{\mathrm{vac}} \, v_{K_6}\|}{\|v_{K_6}\|} = 23.55\%.
\end{equation}
The projection is \emph{entirely} in the $\rho_1$ component:
\begin{equation}\label{eq:pure_rho1}
  \Pi_{\mathrm{vac}} \, v_{K_6} \in V_{\rho_1}, \qquad
  \text{with } V_{\rho_2} \text{ and } V_{\rho_3} \text{ components exactly zero.}
\end{equation}
\end{theorem}

More strongly:

\begin{theorem}[$\rho_2$ inaccessibility]\label{thm:rho2}
No $K_6$ eigenvector (at \emph{any} eigenvalue) projects onto $V_{\rho_2}$.
The hub sector of $K_8$ accesses only $\rho_1$ and $\rho_3$:
\begin{center}
\begin{tabular}{lcc}
  \toprule
  \textbf{$K_6$ eigenvalue} & \textbf{$\Z_7$ sector} & \textbf{Yukawa rank} \\
  \midrule
  $\lambda = 2.764$ (vacuum) & $\rho_1$ (100\%) & 3 \\
  $\lambda = 4.000$ & $\rho_3$ (100\%) & 2 \\
  $\lambda = 4.764$ & $\rho_1$ (100\%) & 3 \\
  $\lambda = 7.236$ & $\rho_1$ (100\%) & 3 \\
  $\lambda = 8.000$ & $\rho_3$ (100\%) & 2 \\
  \bottomrule
\end{tabular}
\end{center}
(Only eigenvectors with nonzero projection onto $V_{\mathrm{vac}}$ are shown.)
\end{theorem}

This establishes $\rho_2$ as structurally inaccessible from the Higgs sector---a proven selection rule, not a fine-tuning.


\subsection{The zero-parameter Yukawa matrix}

Using the $K_6$-determined direction in $V_{\rho_1}$, the $3 \times 3$ Yukawa matrix is:
\begin{equation}\label{eq:Y3}
  Y_3 = \begin{pmatrix}
    -0.298 & 0 & -0.072 \\
    0.113 & 0 & 0.258 \\
    0.377 & -0.074 & -0.377
  \end{pmatrix},
\end{equation}
giving:
\begin{equation}\label{eq:hierarchy}
  Y^\dagger Y \text{ eigenvalues: } 0.6876, \; 0.1122, \; 0.0008,
  \qquad \text{ratio } \mathbf{415 : 135 : 1}.
\end{equation}

This is the main result: \textbf{a zero-parameter prediction of a rank-3 Yukawa with three-generation hierarchy from purely combinatorial-topological data.}

The middle eigenvalue ratio $135:1$ is within 1.5\% of the Standard Model $m_c^2/m_u^2 \approx 137$.


% ============================================================================
% SECTION 7: THE VACUUM MODULI SPACE
% ============================================================================
\section{The vacuum moduli space}\label{sec:moduli}

\subsection{Three sectors from three irreps}

The six-dimensional vacuum eigenspace parametrizes a family of Yukawa matrices.
Each $\rho_k$ subspace is two-dimensional over $\R$, so a direction within it is specified by a single angle $\theta_k \in [0, \pi)$.
The full moduli space is:
\begin{equation}
  \mathcal{M}_{\mathrm{vac}} = (S^1/\Z_2)^3,
  \qquad \text{three independent angles.}
\end{equation}
The Standard Model has three independent Yukawa matrices (up-type, down-type, charged lepton), each specified by one ``strength'' parameter controlling its hierarchy.
The correspondence $|\mathcal{M}_{\mathrm{vac}}| = 3 = |\{\text{SM sectors}\}|$ is a consequence of $\Z_7$ representation theory, not a model-building choice.

\subsection{Hierarchy distribution}\label{sec:distribution}

Scanning 10,000 random unit vectors in $V_{\mathrm{vac}}$:

\begin{itemize}
  \item \textbf{Rank 3} is achieved in 9,999 of 10,000 trials (99.99\%).
  Three massive generations is the \emph{generic} outcome.
  \item The hierarchy $\lambda_1/\lambda_3$ (largest to smallest $Y^\dagger Y$ eigenvalue) ranges from 2 to $3.8 \times 10^9$, with median~173.
  \item The distribution peaks at $10^2$--$10^3$, precisely the Standard Model range.
\end{itemize}

The best matches to Standard Model fermion mass-squared ratios within the vacuum eigenspace are:
\begin{center}
\begin{tabular}{lccc}
  \toprule
  \textbf{Sector} & \textbf{SM target} & \textbf{$K_8$ best match} & \textbf{Accuracy} \\
  \midrule
  Down-type & $790 : 20 : 1$ & $768 : 20 : 1$ & 3\% / 1\% \\
  Up-type & $18{,}700 : 137 : 1$ & $12{,}377 : 153 : 1$ & 34\% / 12\% \\
  Lepton & $12{,}100 : 43 : 1$ & $11{,}893 : 68 : 1$ & 2\% / 58\% \\
  \bottomrule
\end{tabular}
\end{center}

The down-type match ($768:20:1$ vs.\ $790:20:1$) is essentially exact.


\subsection{Doublet position dependence}\label{sec:doublet}

The $\Z_7$ symmetry of the Gram matrix means the doublet position $\{i, i+1 \pmod{7}\}$ can be shifted without changing the vacuum eigenvalue.
However, the Yukawa matrix depends on which vertices are doublet versus generation.
The seven positions give:

\begin{center}
\begin{tabular}{lcc}
  \toprule
  \textbf{Doublet} & \textbf{Hierarchy} & \textbf{Rank} \\
  \midrule
  $\{0,1\}$ & $415 : 135 : 1$ & 3 \\
  $\{1,2\}$ & $544 : 74 : 1$ & 3 \\
  $\{2,3\}$ & $25 : 2 : 1$ & 3 \\
  $\{3,4\}$ & $7 : 1$ & 2 \\
  $\{4,5\}$ & $420 : 109 : 1$ & 3 \\
  $\{5,6\}$ & $18 : 2 : 1$ & 3 \\
  $\{6,0\}$ & $54 : 1$ & 2 \\
  \bottomrule
\end{tabular}
\end{center}

Doublets adjacent to vertex~7 (the handle vertex)---specifically $\{3,4\}$ and $\{6,0\}$ (which becomes $\{6,7\}$ upon relabeling)---give rank~2, consistent with the handle edge being in the doublet coupling rather than a spectator.
Doublets distant from vertex~7 give rank~3 with large hierarchies.


% ============================================================================
% SECTION 8: PHYSICAL INTERPRETATION
% ============================================================================
\section{Physical interpretation}\label{sec:physics}

\subsection{Three sectors: charged fermions and neutrinos}

The $\Z_7$ selection rules partition the vacuum into three sectors with distinct physical properties:

\begin{center}
\begin{tabular}{lcccl}
  \toprule
  \textbf{Irrep} & \textbf{$K_6$ access} & \textbf{Rank} & \textbf{Hierarchy} & \textbf{Interpretation} \\
  \midrule
  $\rho_1$ & Vacuum & 3 & $415:135:1$ & Charged fermion (Higgs-coupled) \\
  $\rho_3$ & Excited & 2 & $\sim 77:1$ & Charged fermion (rank-deficient) \\
  $\rho_2$ & \textbf{None} & --- & Decoupled & Neutrino sector (non-Higgs mass) \\
  \bottomrule
\end{tabular}
\end{center}

In the Standard Model, charged fermion masses arise from Yukawa couplings to the Higgs field, while neutrino masses likely have a different origin (Majorana mass, seesaw mechanism, or higher-dimensional operators).
The $\rho_2$ inaccessibility from the Higgs sector (Theorem~\ref{thm:rho2}) provides a \emph{geometric} explanation for this qualitative difference: the graph topology of $K_8$ forbids one $\Z_7$ sector from coupling to the $K_6$ vacuum at any excitation level.


\subsection{Custodial symmetry}

The vacuum dimension $\dim V_{\mathrm{vac}} = 6$ coincides with $\dim\,\so(4) = \dim(\su(2)_L \times \su(2)_R)$, the custodial symmetry algebra of the Higgs sector.
The decomposition into three two-dimensional irrep spaces corresponds to the three independent $\su(2)$ rotations that parametrize the relative orientation of left- and right-handed gauge sectors.
Whether this numerological coincidence reflects a deeper connection between $\Z_7$ representation theory and custodial symmetry is an open question.


\subsection{Comparison with \texorpdfstring{$K_6$}{K6}}

\begin{center}
\begin{tabular}{lcc}
  \toprule
  \textbf{Property} & \textbf{$K_6$ on $T^2$} & \textbf{$K_8$ on $\Sigma_2$} \\
  \midrule
  Observable & $m_H$ (Higgs mass) & $Y_{\alpha\beta}$ (Yukawa matrix) \\
  Spectral action sector & Bosonic ($a_4/a_2^2$) & Fermionic (vacuum eigenvectors) \\
  Vacuum & Unique ($\lambda = 3.306$) & 6-fold degenerate ($\lambda = 1.960$) \\
  Yukawa rank & 2 & 3 \\
  Hierarchy & $\sim 64:1$ & $415:135:1$ \\
  Free parameters & 0 & 0 ($K_6$ vacuum fixes direction) \\
  \bottomrule
\end{tabular}
\end{center}

$K_6$ computes the Higgs mass (a bulk spectral invariant).
$K_8$ computes the Yukawa structure (a directional vacuum-dependent observable).
They are complementary sectors of a single framework, connected by the embedding $K_6 \hookrightarrow K_7 \hookrightarrow K_8$.


% ============================================================================
% SECTION 9: ALGEBRAIC IDENTITIES
% ============================================================================
\section{Algebraic identities}\label{sec:algebra}

Several structural relations constrain the framework independent of specific computations.

\begin{proposition}[$K_8 = K_6 \otimes \Z_7$]\label{prop:tensor}
The 105 matchings of $K_8$ decompose into 15 orbits of 7 under $\Z_7$:
\begin{equation}
  105 = 7 \times 15 = |\Z_7| \times |M(K_6)|.
\end{equation}
Each orbit contains one representative matching whose hub extension (adding edge $(0,1)$) corresponds to a $K_6$ matching.
\end{proposition}

\begin{proposition}[Handle gauge redundancy]\label{prop:gauge}
For any phase assignment $\zeta_{d_3}$ to the handle direction, the Yukawa coupling $Y^\dagger Y$ is independent of $\zeta_{d_3}$.
\end{proposition}

\begin{proof}
By Lemma~\ref{lem:handle}, every matching $M_i$ contains exactly one handle edge, so $\zeta_i \to \zeta_{d_3} \cdot \zeta_i$ for all~$i$.
In $Y_{\alpha\beta} = \sum_M v_M \zeta_M$, this gives $Y \to \zeta_{d_3} Y$, hence $Y^\dagger Y \to |\zeta_{d_3}|^2 Y^\dagger Y = Y^\dagger Y$.
\end{proof}

\begin{proposition}[Symmetry breaking is mandatory]\label{prop:breaking}
The trivial representation $\rho_0$ of $\Z_7$ does not appear in $V_{\mathrm{vac}}$.
Every vacuum state breaks $\Z_7$.
\end{proposition}


% ============================================================================
% SECTION 10: DISCUSSION AND OPEN QUESTIONS
% ============================================================================
\section{Discussion and open questions}\label{sec:discussion}

\subsection{What is determined}

The framework produces the following without free parameters:
\begin{enumerate}
  \item Three massive fermion generations (generic, 99.99\% of vacuum directions).
  \item Hierarchy scale $10^2$--$10^3$ (peak of distribution matches SM range).
  \item One $\Z_7$ sector decoupled from Higgs ($\rho_2$ selection rule).
  \item $K_6$ vacuum selects $\rho_1$ uniquely, giving hierarchy $415:135:1$.
  \item Rank-2 Yukawa for $\rho_3$ sector (accessible only from excited $K_6$ states).
\end{enumerate}

\subsection{Spectral sum over eigenspaces}

The $K_6$ vacuum projects onto all $K_8$ eigenspaces, not only the vacuum:
\begin{center}
\begin{tabular}{cc}
  \toprule
  \textbf{$K_8$ eigenvalue} & \textbf{$K_6$ vacuum weight} \\
  \midrule
  $\lambda = 1.960$ (vacuum) & 6.0\% \\
  $\lambda = 3.286$ & 28.5\% (largest) \\
  $\lambda = 2.616$ & 11.5\% \\
  other & 54.0\% \\
  \bottomrule
\end{tabular}
\end{center}

The full spectral Yukawa should sum over all eigenspaces weighted by a spectral function $f(\lambda)$.
This would modify the $415:135:1$ prediction, potentially bringing it closer to specific SM sectors.
This computation requires specifying the spectral function, which introduces theoretical dependence beyond pure combinatorics.

\subsection{What activates \texorpdfstring{$\rho_2$}{rho2}?}

If $\rho_2$ represents the neutrino sector, its mass must arise from a mechanism other than the Higgs VEV.
Candidates include:
\begin{itemize}
  \item A Majorana mass term from the genus-2 topology (the second handle provides a $\Z_2$ involution that could generate a Majorana coupling).
  \item A seesaw mechanism where the $\rho_2$ sector couples to a heavy right-handed neutrino field encoded in the non-hub matching structure.
  \item Higher-order spectral action terms (beyond $a_2$ and $a_4$) that break the selection rule.
\end{itemize}

\subsection{CKM and PMNS mixing}

The seven doublet positions in Section~\ref{sec:doublet} produce different mass patterns from the same vacuum.
Mixing between these ``flavor frames'' could encode the CKM and PMNS matrices.
The $\Z_7$ group structure provides a natural framework for discrete flavor symmetries, connecting to the extensive literature on $\Z_N$ and $\Delta(27)$ flavor models~\cite{Feruglio2019,King2017}.


\subsection{\texorpdfstring{$K_8$}{K8} Higgs mass}

An important consistency check: does the $K_8$ spectral action ($a_2$ and $a_4$ from the $105 \times 105$ system) reproduce $m_H \approx 125$~GeV?
If yes, $K_8$ supersedes $K_6$ as the fundamental object.
If no, $K_6$ and $K_8$ genuinely compute different sectors (bosonic vs.\ fermionic), consistent with the Seeley--DeWitt coefficient interpretation.

% ============================================================================
% SECTION 11: CONCLUSIONS
% ============================================================================
\section{Conclusions}\label{sec:conclusions}

We have shown that the complete graph $K_8$, embedded on a genus-2 surface via the Heawood direction structure, produces a three-generation Yukawa coupling with Standard Model-scale mass hierarchy from zero free parameters.
The principal findings:

\begin{enumerate}
  \item \textbf{Direction structure from topology.}
  The 28 edges partition into four classes of seven: three from the $\Z_7$-symmetric Heawood map of $K_7$ on the torus, plus seven handle edges connecting the genus-raising vertex.
  Every matching contains exactly one handle edge, making the handle phase gauge-redundant.

  \item \textbf{Six-dimensional vacuum eigenspace.}
  The vacuum at $\lambda = 1.960$ decomposes as $\rho_1 \oplus \rho_2 \oplus \rho_3$ under $\Z_7$, with no trivial component.
  The symmetry must break.

  \item \textbf{$K_6$ determines the Yukawa.}
  The physical Higgs vacuum from $K_6$ projects purely into $\rho_1$, fixing the Yukawa matrix with no free parameters.
  The result is a rank-3 hierarchy $415:135:1$.

  \item \textbf{Selection rule.}
  No $K_6$ eigenvector reaches $\rho_2$, geometrically decoupling one fermion sector from the Higgs mechanism.

  \item \textbf{Standard Model compatibility.}
  The down-type quark mass ratios are reproduced within 3\% from the vacuum moduli space.
  The middle eigenvalue ratio $135:1$ matches $m_c^2/m_u^2 \approx 137$ within 1.5\%.
\end{enumerate}

The progression $K_4 \to K_6 \to K_8$---sphere, torus, double torus---produces signature, Higgs mass, and fermion masses respectively.
Each step adds a topological handle that unlocks a new sector of the Standard Model.
Whether the sequence continues to $K_9$ (genus~3) or $K_{10}$ (genus~4) to capture mixing angles and CP violation remains to be seen.

The framework makes a sharp structural prediction: one of three fermion sectors is Higgs-decoupled, accessible only through non-standard mass mechanisms.
If the Standard Model's neutrino sector corresponds to this decoupled $\rho_2$ irrep, the qualitative difference between charged fermion and neutrino mass generation has a purely topological origin.


% ============================================================================
% REFERENCES
% ============================================================================
\begin{thebibliography}{99}

\bibitem{Chamseddine2007}
A.~H.~Chamseddine, A.~Connes, and M.~Marcolli,
``Gravity and the standard model with neutrino mixing,''
\textit{Adv.\ Theor.\ Math.\ Phys.} \textbf{11}, 991 (2007).

\bibitem{Connes1996}
A.~Connes,
``Gravity coupled with matter and the foundation of non-commutative geometry,''
\textit{Commun.\ Math.\ Phys.} \textbf{182}, 155 (1996).

\bibitem{CCM}
A.~H.~Chamseddine, A.~Connes, and W.~D.~van~Suijlekom,
``Beyond the spectral Standard Model: emergence of Pati--Salam unification,''
\textit{JHEP} \textbf{2013}, 132 (2013).

\bibitem{PaperI}
B.~Porter,
``Emergent Lorentzian geometry from relational antisymmetry on $K_4$,''
(2026).

\bibitem{PaperII}
B.~Porter,
``Cosmic birefringence and hemispherical asymmetry from axial torsion on $K_4$,''
(2026).

\bibitem{Ringel1968}
G.~Ringel and J.~W.~T.~Youngs,
``Solution of the Heawood map-coloring problem,''
\textit{Proc.\ Natl.\ Acad.\ Sci.\ USA} \textbf{60}, 438 (1968).

\bibitem{Feruglio2019}
F.~Feruglio and A.~Romanino,
``Neutrino flavour symmetries,''
\textit{Rev.\ Mod.\ Phys.} (2019), arXiv:1912.06028.

\bibitem{King2017}
S.~F.~King,
``Unified models of neutrinos, flavour and CP violation,''
\textit{Prog.\ Part.\ Nucl.\ Phys.} \textbf{94}, 217 (2017).

\end{thebibliography}

\end{document}
