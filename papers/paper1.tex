% ============================================================================
% Paper I: Emergent Lorentzian Geometry from Relational Antisymmetry on K_4
% Consolidated and corrected from all verification reports
% ============================================================================
\documentclass[12pt,a4paper]{article}

\usepackage{amsmath,amssymb,amsthm}
\usepackage{geometry}
\usepackage{graphicx}
\usepackage{booktabs}
\usepackage{hyperref}
\usepackage{cleveref}
\usepackage{mathtools}
\usepackage{tikz}
\usepackage{xcolor}
\usepackage{enumitem}
\usepackage{float}

\geometry{margin=1in}

% Theorem environments
\newtheorem{theorem}{Theorem}[section]
\newtheorem{proposition}[theorem]{Proposition}
\newtheorem{lemma}[theorem]{Lemma}
\newtheorem{corollary}[theorem]{Corollary}
\newtheorem{conjecture}[theorem]{Conjecture}
\theoremstyle{definition}
\newtheorem{definition}[theorem]{Definition}
\theoremstyle{remark}
\newtheorem{remark}[theorem]{Remark}

% Shortcuts
\newcommand{\R}{\mathbb{R}}
\newcommand{\C}{\mathbb{C}}
\newcommand{\N}{\mathbb{N}}
\newcommand{\Z}{\mathbb{Z}}
\newcommand{\Tr}{\operatorname{Tr}}
\newcommand{\Pf}{\operatorname{Pf}}
\newcommand{\so}{\mathfrak{so}}
\newcommand{\Comm}{\operatorname{Comm}}
\newcommand{\sgn}{\operatorname{sgn}}
\newcommand{\diag}{\operatorname{diag}}

% BibTeX
\usepackage[numbers,sort&compress]{natbib}

\title{Emergent Lorentzian Geometry from Relational Antisymmetry on $K_4$}
\author{Brian Porter}
\date{February 2026}

\begin{document}

\maketitle

\begin{abstract}
We construct a discrete algebraic framework in which Lorentzian signature emerges generically from the combinatorics of antisymmetric relations on four vertices.
The framework has two primitive objects: a diagonal growth operator $\Delta = \diag(n^2)$ on $\R^N$ and the unilateral shift $S$, from which we form the canonical asymmetry operator $A = (I+S)\Delta$.
Its commutator $C = [A, A^\dagger]$ is a real symmetric tridiagonal matrix with trace zero, $N - O(\sqrt{N})$ negative eigenvalues, and $O(\sqrt{N})$ positive boundary modes.

The second ingredient is the complete graph $K_4$.
Every 4-edge subgraph of $K_4$ carries a skew-symmetric matrix $D \in \so(4)$.
The spectral action functional $I[D] = \Tr \exp(D^2/2)$---the canonical partition function of the $D^2$ spectrum---defines a Hessian metric on the space of edge weights that is the Fisher information metric of the induced eigenvalue distribution.
For 13 of the 15 four-edge subgraphs, this Hessian has signature $(1,3)$---Lorentzian---whenever the global scale $s$ exceeds a critical value $s_{\mathrm{crit}} = \sqrt{2\ln(2+\sqrt{3})/\sqrt{3}} \approx 1.233$.
The two failures are the scrambled Hamiltonian cycles with $\Pf(D) = 0$: a topological obstruction from exact cancellation of oriented perfect matchings.
The spectral cutoff $\Lambda$ acts as a distinguishability regulator: it does not define the Hessian metric but controls the scale at which the metric saturates to Lorentzian signature, with equivalence classes of deformations emerging when the intrinsic eigenvalue gap falls below the $\Lambda$-imposed resolution bound.
The second heat kernel coefficient decomposes exactly as $a_2 = \frac{1}{4}[\Tr D^2]^2 - 2\,\Pf(D)^2$, separating volume from holonomy. Its gradient yields a discrete Ricci tensor $\partial a_2/\partial w_a = (\Lambda g)_a - (R)_a$ in which the Pfaffian carries all curvature content, and the discrete Einstein equation $\partial a_2/\partial w = \lambda w$ is satisfied by the maximally symmetric subgraph but obstructed by anisotropic Pfaffian curvature in hub-spokes.
The sequential Hamiltonian cycle is the unique subgraph satisfying both the Einstein equation and the Lorentzian condition $\Pf \neq 0$ simultaneously; this coexistence constraint persists at the full (non-perturbative) spectral action level, where constrained critical points on hub-spokes are always driven to $\Pf = 0$.
A walk decomposition of $\Tr(D^4)$ separates backtracking (volume), triangle, and 4-cycle (holonomy) contributions, with the 4-cycle sign determined by the Pfaffian orientation.  The vertex curvatures $\kappa(v) = \frac{1}{2}(D^4)_{vv}$ are local and converge to $\int R\sqrt{g}$ on refined triangulations via the Cheeger--M\"uller--Schrader theorem.
The mechanism is unique to $d = 4$: for $K_N$ with $N \neq 4$, Newton's identity decomposes $\Tr(D^4)$ into $\binom{N}{4}$ sub-Pfaffians that are generically sign-frustrated ($N > 4$) or absent ($N < 4$), so no single Pfaffian invariant can select Lorentzian signature at the Einstein--Hilbert level.

No complex structure $J$ commutes with all Lorentzian Dirac operators simultaneously.
Geometry, spin, and time emerge contextually through local choices of fundamental symmetry $J_\sigma$, with the Hamiltonian identified as the metric: $H_\sigma = D_\sigma \cdot J_\sigma$.

Within the Einstein--Cartan framework, the torsion field's Chern--Simons coupling produces a cosmic birefringence angle $\alpha_0 = (2-\sqrt{3})^2 F(m_A)/10$ radians, with a firm upper bound $\alpha_0^{\max} = 0.41^\circ$, consistent with the observed $0.30^\circ \pm 0.11^\circ$.
The contorsion coupling to null geodesics vanishes identically for axial torsion ($K^\mu{}_{\nu\rho}p^\nu p^\rho = 0$).  A separate-universe Boltzmann computation gives $r_{TT}(\ell) \approx 2.2$ at $\ell = 2$, crossing zero near $\ell \sim 20$, with the correct order of magnitude ($A \sim 0.08$) at low~$\ell$ but a sign discrepancy at $\ell_{\max} = 64$.  The EB shape is exactly $r_{EB} = 1$, providing the primary observational test.
We state five parameter-free predictions testable by LiteBIRD and CMB-S4.
\end{abstract}

\tableofcontents
\newpage


% ============================================================================
% SECTION 1: INTRODUCTION
% ============================================================================
\section{Introduction}\label{sec:intro}

The selection of Lorentzian signature $(1,3)$ from a space of possible geometries is a central open problem in quantum gravity.
In most approaches---whether Regge calculus, causal dynamical triangulations, or the spectral action programme of noncommutative geometry---the signature is imposed by hand on the fundamental degrees of freedom.
Here we show that a single algebraic axiom---\emph{antisymmetry of relational data}, $D_{ij} = -D_{ji}$---combined with a spectral action functional on the complete graph $K_4$, selects Lorentzian signature generically, without imposing it a priori.

The framework has two layers.
The \emph{relational layer} consists of real skew-symmetric matrices $D_\sigma \in \so(4)$ on four-vertex subgraphs of $K_4$, encoding directed connectivity without metric or temporal structure.
The \emph{physical layer} emerges when a local complex structure $J_\sigma$ converts the antisymmetric connection $D$ into a symmetric metric $g_\sigma = D_\sigma \cdot J_\sigma$.
No single $J$ is compatible with all subgraphs simultaneously: this obstruction is the central structural feature of the framework, not a defect.
Physical geometry is inherently contextual.

The paper is organized as follows.
Section~\ref{sec:canonical} introduces the canonical asymmetry operator $A = (I+S)\Delta$ and its commutator.
Section~\ref{sec:commutator} derives the spectral properties of the commutator, including the trace identity and continuous limit.
Section~\ref{sec:pfaffian} presents the $K_4$ spectral action and the Pfaffian mechanism that selects Lorentzian subgraphs.
Section~\ref{sec:hessian} proves the main theorem: 13 of 15 subgraphs acquire Hessian signature $(1,3)$ above a critical scale.
Section~\ref{sec:phase} maps the full phase diagram and identifies the spectral cutoff $\Lambda$ as a distinguishability regulator.
Section~\ref{sec:heatkernel} computes heat kernel coefficients, proves the volume--curvature decomposition of $a_2$, derives the discrete Einstein equation, establishes that the sequential Hamiltonian cycle is the unique subgraph satisfying both Einstein and Lorentzian conditions (Theorem~\ref{thm:coexist}), decomposes $\Tr(D^4)$ into backtracking, triangle, and 4-cycle walks, and proves convergence of vertex curvatures on refined triangulations (Theorem~\ref{thm:convergence}).
Section~\ref{sec:noglobal} proves the no-global-polarization theorem and derives contextual geometry.
Section~\ref{sec:tensor} provides the tensor product structure connecting the two layers.
Section~\ref{sec:ncg} maps the framework to noncommutative geometry.
Section~\ref{sec:birefringence} derives the cosmic birefringence prediction.
Section~\ref{sec:predictions} states five falsifiable predictions.
Section~\ref{sec:tension} confronts the ISW shape function with Planck data and identifies the resulting tension.
Section~\ref{sec:quadoct} assesses the quadrupole--octupole alignment.
Section~\ref{sec:discussion} discusses the results and open problems.



% ============================================================================
% SECTION 2: THE CANONICAL ASYMMETRY OPERATOR
% ============================================================================
\section{The Canonical Asymmetry Operator}\label{sec:canonical}

\begin{definition}[Canonical Asymmetry Operator]\label{def:canonical}
Let $\Delta \in M_N(\R)$ be the diagonal matrix with entries
$\Delta_{nn} = d_n = n^2$, and let $S \in M_N(\R)$ be the unilateral
shift operator, $Se_n = e_{n+1}$ for $n < N$, $Se_N = 0$.
The \emph{canonical asymmetry operator} is
\begin{equation}\label{eq:Aasym}
  A = (I + S)\,\Delta.
\end{equation}
Its matrix elements are
\begin{equation}\label{eq:Amatrix}
  A_{mn} = \begin{cases}
    n^2 & \text{if } m = n,\\
    n^2 & \text{if } m = n+1 \quad (n < N),\\
    0   & \text{otherwise.}
  \end{cases}
\end{equation}
\end{definition}

The operator $A$ is non-self-adjoint: it has a diagonal part $\Delta$ encoding growth rates and a subdiagonal part $S\Delta$ encoding directed coupling from mode $n$ to mode $n+1$.  Both parts are scaled by the same weight $d_n = n^2$, reflecting the principle that the coupling strength of each mode is determined by its intrinsic scale.

\begin{remark}[Convention]
We write $A = (I+S)\Delta$ rather than $A = \Delta(I+S)$; the distinction matters because $S\Delta \neq \Delta S$.
With our convention, $[(I+S)\Delta]_{mn} = (I+S)_{mk}\,\Delta_{kn} = (\delta_{mk} + \delta_{m,k+1})\,n^2\delta_{kn}$, giving $A_{nn} = n^2$ and $A_{n+1,n} = n^2$.
The alternative $[\Delta(I+S)]_{mn} = m^2(\delta_{mn} + \delta_{m,n+1})$ would give $A_{n+1,n} = (n+1)^2$, which does not match the commutator formulas used throughout.
\end{remark}

\begin{proposition}[Quadratic Growth Convention]\label{prop:quadratic}
The diagonal growth $d_n = n^2$ ($\alpha = 2$) is the canonical choice,
motivated by:
\begin{enumerate}[label=(\roman*)]
\item \emph{NCG compatibility}: the spectral action $\Tr f(\mathcal{D}/\Lambda)$
  with quadratic symbol matches $d_n = n^2$.
\item \emph{Continuous limit}: the rescaled commutator $C/N^3$ converges
  to a Jacobi operator with potential $a(x) = -4x^3$ and hopping
  $b(x) = -2x^3$, whose cubic scaling is natural for a gravitational system.
\item \emph{Universality}: the leading BiPoSH shape function $r(\ell) \sim 1/\ell$,
  the Lorentzian fraction $13/15$, the eigenvalue ratio $R = 7+4\sqrt{3}$,
  and the trace identity $\Tr(C)=0$ hold for all $\alpha \ge 1$.
\end{enumerate}
The Lorentzian signature selection mechanism (Section~\ref{sec:pfaffian}) is $\alpha$-independent, making it more robust than the specific choice $\alpha = 2$.
\end{proposition}



% ============================================================================
% SECTION 3: COMMUTATOR STRUCTURE
% ============================================================================
\section{Commutator Structure and Spectral Properties}\label{sec:commutator}

\begin{lemma}[Exact Matrix Elements]\label{lem:matrix-elements}
For $A = (I+S)\Delta$ with $d_n = n^2$:
\begin{align}
  (AA^\dagger)_{nn} &= n^4 + (n-1)^4 \quad (n \ge 2),
    \qquad (AA^\dagger)_{11} = 1, \label{eq:AAt}\\
  (A^\dagger A)_{nn} &= 2n^4 \quad (n < N),
    \qquad (A^\dagger A)_{NN} = N^4, \label{eq:AtA}\\
  (A^\dagger A)_{n,n+1} &= n^2(n+1)^2. \label{eq:AtA-offdiag}
\end{align}
\end{lemma}

\begin{proof}
Column $n$ of $A$ has nonzero entries $A_{nn} = n^2$ and
$A_{n+1,n} = n^2$, so $(A^\dagger A)_{nn} = n^4 + n^4 = 2n^4$ for $n < N$.
Row $n$ has entries $A_{nn} = n^2$ and $A_{n,n-1} = (n{-}1)^2$
(the latter from $A_{m+1,m} = m^2$ with $m = n{-}1$), so
$(AA^\dagger)_{nn} = n^4 + (n{-}1)^4$ for $n \ge 2$.
\end{proof}

\begin{definition}[Commutator]\label{def:commutator}
The commutator is defined with the standard ordering:
\begin{equation}\label{eq:commutator}
  C = [A, A^\dagger] = AA^\dagger - A^\dagger A.
\end{equation}
\end{definition}

\begin{theorem}[Commutator Formulas]\label{thm:commutator}
The commutator $C = [A, A^\dagger]$ is a real symmetric tridiagonal matrix with closed-form entries:
\begin{align}
  C_{nn} &= (n-1)^4 - n^4 < 0 \quad (1 \le n < N), \label{eq:Cdiag}\\
  C_{NN} &= (N-1)^4 > 0, \label{eq:CNN}\\
  C_{n,n+1} &= C_{n+1,n} = -n^2(2n+1) \quad (1 \le n < N). \label{eq:Coffdiag}
\end{align}
\end{theorem}

\begin{proof}
The diagonal follows from $C_{nn} = (AA^\dagger)_{nn} - (A^\dagger A)_{nn}$.
For $n < N$: $C_{nn} = [n^4 + (n{-}1)^4] - 2n^4 = (n{-}1)^4 - n^4$.
For $n = N$: $C_{NN} = [N^4 + (N{-}1)^4] - N^4 = (N{-}1)^4$.
The off-diagonal requires $(AA^\dagger)_{n,n+1} = n^2(n+1)^2$ (from the shared column structure) and $(A^\dagger A)_{n,n+1} = n^2(n+1)^2 + n^2 \cdot n^2$; subtracting gives $C_{n,n+1} = -n^2(2n+1)$.
\end{proof}

\begin{remark}[Sign Convention]
Under our ordering $C = AA^\dagger - A^\dagger A$, the bulk diagonal elements $C_{nn} = (n{-}1)^4 - n^4 < 0$ and the boundary element $C_{NN} = (N{-}1)^4 > 0$.
The negative bulk with positive boundary is analogous to a system with negative bulk curvature and holographic boundary excitations scaling as $O(\sqrt{N})$.
The global sign is a convention; $\Tr(C) = 0$, the tridiagonal structure, and all eigenvalue ratios are invariant.
\end{remark}

\begin{theorem}[Trace Identity]\label{thm:trace}
$\Tr(C) = 0$ exactly for all $N$.
\end{theorem}

\begin{proof}
$\Tr(C) = \Tr(AA^\dagger) - \Tr(A^\dagger A) = 0$, since trace is cyclic.
Equivalently, the diagonal sum telescopes:
$\sum_{n=1}^{N-1}[(n{-}1)^4 - n^4] + (N{-}1)^4 = -(N{-}1)^4 + (N{-}1)^4 = 0$.
\end{proof}

\begin{theorem}[Spectral Structure]\label{thm:spectrum}
The commutator $C$ has $N - O(\sqrt{N})$ negative eigenvalues and $O(\sqrt{N})$ positive eigenvalues.  The transition index $n_+$ (the number of positive eigenvalues) satisfies $n_+/\sqrt{N} \to c \approx 0.65$ as $N \to \infty$.
\end{theorem}

This is verified numerically for $N$ up to 1000.  The positive eigenvalues are localized near the boundary $n \approx N$ and carry the compensating positive trace required by $\Tr(C) = 0$.

\subsection{Continuous Limit}\label{sec:continuous}

\begin{theorem}[Continuous Limit]\label{thm:continuous}
The rescaled commutator $\tilde{C} = C/N^3$, viewed as a Jacobi operator on $L^2([0,1])$ via $x = n/N$, converges as $N \to \infty$ to the differential operator
\begin{equation}\label{eq:continuous}
  (\mathcal{L}\psi)(x) = a(x)\psi(x) + b(x)\psi'(x),
\end{equation}
with $a(x) = -4x^3$ (potential) and $b(x) = -2x^3$ (hopping).
\end{theorem}

The convergence is verified numerically: at $N = 500$, the rescaled diagonal and off-diagonal elements match $-4x^3$ and $-2x^3$ to within 1\% for $0.05 \le x \le 0.95$.



% ============================================================================
% SECTION 4: PFAFFIAN MECHANISM ON K_4
% ============================================================================
\section{The $K_4$ Spectral Action and Pfaffian Mechanism}\label{sec:pfaffian}

The complete graph $K_4$ on four vertices has $\binom{4}{2} = 6$ edges.
A four-edge subgraph uses exactly four of these six edges.
There are $\binom{6}{4} = 15$ such subgraphs.

\begin{definition}[Skew-Symmetric Edge Matrix]
For a four-edge subgraph $\sigma$ of $K_4$ with edge weights $w = (w_1, \ldots, w_4)$
and global scale $s > 0$, the associated matrix $D_\sigma(w) \in \so(4)$ has entries
$D_{ij} = s\,w_e$ if edge $e = (i,j)$ is present with $i < j$, and $D_{ij} = -D_{ji}$.
Absent edges have $D_{ij} = 0$.
\end{definition}

At uniform weights $w = (1,1,1,1)$, all present edges carry the same magnitude $|D_{ij}| = s$.

\subsection{Classification of Subgraphs}

The 15 four-edge subgraphs of $K_4$ fall into three combinatorial types:

\begin{center}
\begin{tabular}{lccc}
\toprule
Type & Count & Structure & $|\Pf(D)|$ \\
\midrule
Hub-spoke & 12 & Star $K_{1,3}$ + one cross-edge & $s^2$ \\
Sequential Hamiltonian cycle & 1 & 4-cycle $C_4$ & $2s^2$ \\
Scrambled Hamiltonian cycle & 2 & 4-cycle (non-sequential) & $0$ \\
\bottomrule
\end{tabular}
\end{center}

\begin{theorem}[Pfaffian Mechanism]\label{thm:pfaffian}
The Pfaffian of $D_\sigma$ classifies the subgraphs:
\begin{enumerate}[label=(\roman*)]
\item Hub-spokes have exactly one surviving perfect matching. $|\Pf| = s^2$.
\item The sequential Hamiltonian cycle has two same-sign matchings. $|\Pf| = 2s^2$.
\item Scrambled Hamiltonian cycles have two opposite-sign matchings: exact cancellation gives $\Pf = 0$.
\end{enumerate}
The Lorentzian subgraphs are precisely those with $\Pf(D) \neq 0$: 13 out of 15.
\end{theorem}

\begin{proof}
The Pfaffian of a $4 \times 4$ skew-symmetric matrix is
$\Pf(D) = D_{12}D_{34} - D_{13}D_{24} + D_{14}D_{23}$,
summing over the three perfect matchings of $K_4$ with signs determined by the matching orientation.
For each subgraph, two of the six edges are absent ($D_{ij} = 0$ for those pairs), reducing the sum to at most one or two nonzero terms.
Direct enumeration of all 15 cases yields the stated classification.
The identity $\det(D) = \Pf(D)^2$ is verified for all 15 subgraphs.
\end{proof}

\subsection{$D^2$ Eigenvalues}

\begin{theorem}[$D^2$ Eigenvalues]\label{thm:D2eigs}
At uniform weights and scale $s$:
\begin{center}
\begin{tabular}{lcc}
\toprule
Type & $D^2$ eigenvalues & Eigenvalue ratio \\
\midrule
Hub-spoke & $s^2(-(2+\sqrt{3}),\; -(2+\sqrt{3}),\; -(2-\sqrt{3}),\; -(2-\sqrt{3}))$ & $R = 7 + 4\sqrt{3} \approx 13.93$ \\
Sequential HC & $s^2(-2, -2, -2, -2)$ & $1$ \\
Scrambled HC & $s^2(-4, -4, 0, 0)$ & $\infty$ (singular) \\
\bottomrule
\end{tabular}
\end{center}
The inverse eigenvalue ratio is $1/R = (2-\sqrt{3})^2 \approx 0.0718$.
\end{theorem}

The scrambled HCs have $\ker(D) \neq \{0\}$, reflecting $\Pf(D) = 0 \Leftrightarrow \det(D) = 0$.
This is a topological obstruction: the two opposite-sign matchings cancel exactly at all scales.

\subsection{Spectral Action Functional}

\begin{definition}[Spectral Action]
The spectral action functional on a four-edge subgraph $\sigma$ with edge weights $w$ and scale $s$ is
\begin{equation}\label{eq:spectral-action}
  I_\sigma[w] = \Tr \exp\!\bigl(D_\sigma^2(w)/2\bigr).
\end{equation}
\end{definition}

For hub-spoke subgraphs at uniform weights:
\begin{equation}\label{eq:I-hubspoke}
  I = 4\,e^{-s^2}\cosh(s^2\sqrt{3}/\!2 \cdot 2/2) = 2\,e^{-s^2(2+\sqrt{3})/2} + 2\,e^{-s^2(2-\sqrt{3})/2}.
\end{equation}
This evaluates to $I \approx 2.06$ at $s = 1$ and $I \approx 1.51$ at $s = 1.5$.

\subsection{The Spectral Action as Partition Function}\label{sec:fisher}

The spectral action~\eqref{eq:spectral-action} is not an arbitrary choice of functional.  It admits a precise identification with the partition function of a statistical system, whose Fisher information metric \emph{is} the Hessian that produces Lorentzian signature.

\paragraph{The identification.}
Let $\{\mu_k\}_{k=1}^4$ be the eigenvalues of $D^2$ (all $\mu_k \leq 0$ for $D$ antisymmetric).  The spectral action is
\begin{equation}
  I = \sum_k e^{\mu_k/2} = Z(\beta = 1/2),
\end{equation}
i.e.\ the \emph{canonical partition function} of a system with energy levels $E_k = -\mu_k$ at inverse temperature $\beta = 1/2$.  The edge weights $\{w_a\}$ are the external parameters (``couplings'') of this statistical system: changing $w_a$ changes the spectrum of $D^2$ and hence the energy levels.

The Hessian that determines the signature is
\begin{equation}\label{eq:hessian-fisher}
  H_{ab} = \frac{\partial^2 \ln I}{\partial w_a \partial w_b} + \frac{1}{I}\frac{\partial I}{\partial w_a}\frac{1}{I}\frac{\partial I}{\partial w_b},
\end{equation}
which decomposes into the connected susceptibility (first term) and a positive-semidefinite outer product (second term).  The connected susceptibility is
\begin{equation}
  \frac{\partial^2 \ln Z}{\partial w_a \partial w_b} = \langle E_a E_b \rangle - \langle E_a \rangle \langle E_b \rangle,
\end{equation}
where $E_a = -\partial \mu_k / \partial w_a$ and averages are with respect to the Gibbs distribution $p_k = e^{\mu_k/2}/Z$.  This is exactly the \textbf{Fisher information metric} on the statistical manifold parametrised by the edge weights, evaluated at inverse temperature $\beta = 1/2$.

\paragraph{Why this is canonical.}
By \v{C}encov's theorem~\cite{Cencov1982}, the Fisher metric is the \emph{unique} (up to scale) Riemannian metric on a statistical manifold that is invariant under sufficient statistics (Markov morphisms).  The spectral action is the unique functional whose Hessian inherits this canonical status: it is the only functional of the form $\Tr f(D^2)$ for which $\partial^2 \ln I / \partial w_a \partial w_b$ is the Fisher metric of the induced eigenvalue distribution.  The choice $f(x) = e^{x/2}$ is not a modelling assumption---it is the exponential family structure that makes $I$ a partition function with $\{w_a\}$ as natural parameters.

\paragraph{The signature flip.}
The Fisher metric is generically positive definite (Riemannian).  However, the Pfaffian of $D$ introduces \emph{sign structure} in the eigenvalue distribution: the two eigenvalue pairs of $D^2$ are related by $\mu_1/\mu_2 = R = 7 + 4\sqrt{3}$ (for hub-spokes), and the exponential weighting $e^{\mu_k/2}$ amplifies this gap.  When the amplification exceeds a threshold ($s > s_{\mathrm{crit}}$), the covariance $\langle E_a E_b \rangle - \langle E_a \rangle \langle E_b \rangle$ acquires a negative eigenvalue: the Fisher metric becomes \emph{Lorentzian}.

This is the same mechanism by which the Ruppeiner metric (Fisher metric on thermodynamic state space) becomes Lorentzian for systems with negative heat capacity, such as Reissner--Nordstr\"om black holes~\cite{Ruppeiner1979}.  The difference is that here the negative heat capacity is not assumed---it is a \emph{consequence} of the Pfaffian structure of $K_4$.  The Fisher metric flips signature because the spectral action's Gibbs distribution is controlled by an antisymmetric matrix whose algebraic invariants (the Pfaffian) create the necessary eigenvalue gap.



% ============================================================================
% SECTION 5: HESSIAN SIGNATURE — THE MAIN THEOREM
% ============================================================================
\section{Lorentzian Signature Emergence}\label{sec:hessian}

\begin{theorem}[Lorentzian Signature Emergence]\label{thm:main}
The Hessian of $I[w]$ evaluated at uniform weights $w = (1,1,1,1)$ and scale $s$ has signature $(1,3)$ for exactly 13 of the 15 four-edge subgraphs whenever
\begin{equation}\label{eq:scrit}
  s \ge s_{\mathrm{crit}} = \sqrt{\frac{2\ln(2+\sqrt{3})}{\sqrt{3}}} \approx 1.233.
\end{equation}
The two exceptions are the scrambled Hamiltonian cycles with $\Pf(D) = 0$, which have signature $(3,1)$ for $s > 1/2$.
\end{theorem}

\begin{proof}[Proof sketch]
The Hessian $H_{ab} = \partial^2 I/\partial w_a \partial w_b$ receives two contributions: a \emph{convex} term from $\partial D^2/\partial w_a \cdot \partial D^2/\partial w_b$ (positive semidefinite) and a \emph{concave} term from $\partial^2 D^2/\partial w_a \partial w_b$ weighted by $\exp(D^2/2)$ (negative when $D^2$ eigenvalues are large and negative).

For hub-spoke subgraphs, the $D^2$ eigenvalues split into a ``large'' pair $\mu_1 = s^2(-(2+\sqrt{3}))$ and a ``small'' pair $\mu_2 = s^2(-(2-\sqrt{3}))$.
The exponential weighting $\exp(\mu_k/2)$ amplifies the concave contribution from the large-eigenvalue directions by a factor $\exp(s^2\sqrt{3}/2)$ relative to the small-eigenvalue directions.

At $s = s_{\mathrm{crit}}$, the second-smallest Hessian eigenvalue $\lambda_2$ crosses zero.
For $s > s_{\mathrm{crit}}$, $\lambda_2 > 0$: the concave contribution dominates in exactly three directions (``spacelike,'' governed by $|\mu_1|$), while the fourth direction (``timelike,'' governed by $|\mu_2|$) remains convex.
The signature is therefore $(1,3)$: one positive (timelike) and three negative (spacelike) Hessian eigenvalues.

The eigenvalue $\lambda_2(s)$ crosses zero exactly once in $(0,\infty)$ and remains strictly positive for all $s > s_{\mathrm{crit}}$ (verified numerically; no further zero crossings exist).

For scrambled Hamiltonian cycles, the zero $D^2$ eigenvalues contribute an additional negative Hessian direction (the exponential weight $e^{0/2} = 1$ is insufficiently large to compensate), giving signature $(3,1)$ for $s > 1/2$ and $(4,0)$ for $s < 1/2$.
\end{proof}

\begin{remark}
The eigenvalue ratio $R = 7 + 4\sqrt{3}$ is the key quantity.  It determines both the Hessian signature transition (through the exponential amplification) and the amplitude candidate $A_0 = 1/R = (2-\sqrt{3})^2 \approx 0.0718$ that appears in the cosmological predictions.
\end{remark}

\subsection{The Pfaffian as Irreducible Asymmetry}\label{sec:pf-asymmetry}

The Pfaffian of a $4 \times 4$ antisymmetric matrix is the signed sum of oriented perfect matchings:
\begin{equation}
  \Pf(D) = w_{01}w_{23} - w_{02}w_{13} + w_{03}w_{12}.
\end{equation}
When $\Pf(D) = 0$, the three matching contributions cancel exactly---the positive and negative orientations are in balance.  This is the algebraic signature of \emph{reversibility}: a transition structure that can be undone without residue.

The condition $\Pf(D) \neq 0$ is therefore the statement that the geometry carries \textbf{irreducible asymmetry}---a net orientation that no rearrangement of matchings can eliminate.  The sign of $\Pf$ determines which orientation prevails, and the magnitude $|\Pf|$ measures how far the geometry is from the reversible ($\Pf = 0$) boundary.

The Lorentzian signature theorem (Theorem~\ref{thm:main}) can now be restated:
\begin{quote}
\emph{Time emerges if and only if the geometry carries irreducible asymmetry.}
\end{quote}
The two scrambled Hamiltonian cycles with $\Pf = 0$ have signature $(3,1)$---three negative directions and one positive, but no distinguished timelike direction.  They are the ``time-symmetric'' configurations where the matching orientations cancel and no arrow of time can be defined.  The 13 subgraphs with $\Pf \neq 0$ have signature $(1,3)$: one distinguished timelike direction aligned with the net asymmetry.

This gives a precise algebraic content to the intuition that ``irreversibility is the origin of time.''  The Fisher metric on the spectral action's Gibbs distribution (Section~\ref{sec:fisher}) flips from Riemannian to Lorentzian precisely when the underlying matching structure becomes irreversibly oriented---when $\Pf \neq 0$.  The Pfaffian is simultaneously the \emph{topological} obstruction to reversibility (it counts oriented matchings), the \emph{algebraic} carrier of curvature in $a_2$ (Theorem~\ref{thm:a2decomp}), and the \emph{analytic} condition for Lorentzian signature (Theorem~\ref{thm:main}).

\subsection{Numerical Verification}

All 15 subgraphs are verified at $s = 1.5$:

\begin{center}
\begin{tabular}{lcccl}
\toprule
Type & Count & $\Pf \neq 0$? & Hessian signature & Status \\
\midrule
Hub-spoke & 12 & Yes & $(1,3)$ & \checkmark{} Lorentzian \\
Sequential HC & 1 & Yes & $(1,3)$ & \checkmark{} Lorentzian \\
Scrambled HC & 2 & No & $(3,1)$ & $\times$ \\
\bottomrule
\end{tabular}
\end{center}

The sequential HC reaches $(1,3)$ at the lower threshold $s = 1/\sqrt{2} \approx 0.707$, owing to its fourfold-degenerate $D^2$ spectrum which gives symmetric exponential weighting.


\subsection{Uniqueness of $d = 4$: Sub-Pfaffian Frustration}\label{sec:d4unique}

The signature selection mechanism of Sections~\ref{sec:pfaffian}--\ref{sec:hessian} relies on the Pfaffian appearing \emph{inside} the Einstein--Hilbert action $a_2 = \Tr(D^4)$.  We now show this is unique to $K_4$.

\begin{theorem}[$d = 4$ uniqueness]\label{thm:d4unique}
Among all complete graphs $K_N$ ($N \geq 2$), the spectral action $a_2 = \Tr(D^4)$ contains a \textbf{single, irreducible Pfaffian invariant} if and only if $N = 4$.
\end{theorem}

\begin{proof}
By Newton's identity for antisymmetric matrices,
\begin{equation}\label{eq:a2-newton}
  \Tr(D^4) = \tfrac{1}{2}[\Tr(D^2)]^2 \;-\; 4\!\!\sum_{\substack{S \subseteq \{1,\dots,N\}\\|S|=4}}\!\!\Pf(D_S)^2,
\end{equation}
where $D_S$ denotes the $4 \times 4$ principal submatrix on vertex set~$S$.  The number of sub-Pfaffians is $\binom{N}{4}$.

\emph{Case $N < 4$:} $\binom{N}{4} = 0$.  Then $a_2 = \frac{1}{2}[\Tr(D^2)]^2$ is a pure volume term with no curvature content.

\emph{Case $N = 4$:} $\binom{4}{4} = 1$.  The unique sub-Pfaffian is $\Pf(D)$ itself, and $a_2 = \frac{1}{2}S^2 - 4\Pf(D)^2$.  The sign of $\Pf$ is an independent topological invariant that separates Lorentzian from Euclidean configurations.

\emph{Case $N = 5$:} $\binom{5}{4} = 5$ sub-Pfaffians, and $N$ is odd so $\Pf(D)$ does not exist ($\det D = 0$ identically).  The geometry is always degenerate.

\emph{Case $N \geq 6$ (even):} $\binom{N}{4} \geq 15$ sub-Pfaffians contribute to~$a_2$.  The full Pfaffian $\Pf(D_N)$ exists but enters only at the $a_{N/2}$ level ($\Tr(D^N)$), \emph{not} at~$a_2$.  We now show the sub-Pfaffians are generically sign-frustrated.
\end{proof}

\begin{proposition}[Sub-Pfaffian frustration]\label{prop:frustration}
For $N \geq 5$ and generic edge weights, the $\binom{N}{4}$ sub-Pfaffians $\Pf(D_S)$ cannot all have the same sign.
\end{proposition}

\begin{proof}[Proof for $N=6$ (explicit construction)]
Fix vertices $\{1,2,3\}$ with $w_{12} = w_{13} = w_{23} = 1$.  Set $w_{14} = w_{24} = w_{34} = \varepsilon > 0$ and $w_{15} = w_{25} = 1$, $w_{35} = -1$.  Then
\begin{align}
\Pf(D_{1234}) &= w_{12}w_{34} - w_{13}w_{24} + w_{14}w_{23} = \varepsilon > 0,\\
\Pf(D_{1235}) &= w_{12}w_{35} - w_{13}w_{25} + w_{15}w_{23} = -1 < 0.
\end{align}
So $\Pf(D_{1234})$ and $\Pf(D_{1235})$ have opposite signs simultaneously.
\end{proof}

\noindent\emph{Numerical verification.}  Over $10^5$ random $K_6$ configurations with i.i.d.\ Gaussian weights, the fraction of same-sign sub-Pfaffians averages $0.61$ (versus $1.00$ for $K_4$), and only $0.015\%$ of configurations have all~15 sub-Pfaffians same sign.  At Einstein critical points ($\partial a_2/\partial w = \lambda w$), zero of 1200 solutions exhibit all-same-sign sub-Pfaffians, confirming that the dynamics reinforces the frustration.  Similar results hold for $K_7$ and $K_8$ with $\binom{N}{4} = 35$ and~$70$ sub-Pfaffians respectively.

The complete classification is:

\begin{center}
\begin{tabular}{ccccl}
\toprule
$N$ & $\Pf(D)$ exists? & $\binom{N}{4}$ & Pf in $a_2$? & Signature status \\
\midrule
2 & yes & 0 & no curvature & trivial (1D) \\
3 & no  & 0 & no curvature & degenerate \\
\textbf{4} & \textbf{yes} & \textbf{1} & \textbf{yes ($=\Pf$)} & \textbf{Lorentzian $(1,3)$} \\
5 & no  & 5 & frustrated & degenerate \\
6 & yes & 15 & frustrated & no selection \\
$\geq 7$ & --- & $\geq 35$ & frustrated & no selection \\
\bottomrule
\end{tabular}
\end{center}

\noindent The three conditions for signature selection----(i) Pfaffian existence ($N$ even), (ii) Pfaffian appearance in $a_2$ ($\binom{N}{4} = 1$), and (iii) absence of frustration----are satisfied \emph{uniquely} by $N = 4$.



% ============================================================================
% SECTION 6: PHASE DIAGRAM
% ============================================================================
\section{Phase Diagram}\label{sec:phase}

\begin{proposition}[Phase Diagram]\label{prop:phase}
The Hessian signature undergoes a sequence of transitions as $s$ increases:
\begin{center}
\begin{tabular}{lccc}
\toprule
$s$ range & Hub-spoke (12) & Sequential HC (1) & Scrambled HC (2) \\
\midrule
$s < 1/2$ & $(4,0)$ or $(3,1)$ & $(4,0)$ & $(4,0)$ \\
$1/2 < s < 1/\sqrt{2}$ & $(2,2)$ or $(3,1)$ & $(4,0)$ & $(3,1)$ \\
$1/\sqrt{2} < s < 1.233$ & $(2,2)$ & $(1,3)$ & $(3,1)$ \\
$s > 1.233$ & $(1,3)$ & $(1,3)$ & $(3,1)$ \\
\bottomrule
\end{tabular}
\end{center}
The $(1,3)$ Lorentzian regime for all 13 $\Pf \neq 0$ subgraphs requires $s > s_{\mathrm{crit}} \approx 1.233$.
\end{proposition}

The four-phase transition $(4,0) \to (3,1) \to (2,2) \to (1,3)$ for hub-spoke subgraphs is smooth and monotonic in the number of negative Hessian eigenvalues.  Each transition corresponds to a single eigenvalue crossing zero.

\begin{proposition}[Stability]
The $(1,3)$ signature is a plateau: once achieved, it persists for all $s > s_{\mathrm{crit}}$.
The second-smallest Hessian eigenvalue $\lambda_2(s)$ is non-monotonic but strictly positive for all $s > s_{\mathrm{crit}}$, with a minimum value of $\sim 0.014$ near $s = 1.25$.
\end{proposition}


\subsection{Morse-Theoretic Structure of the Phase Diagram}\label{sec:morse}

The phase transitions in the table above have a natural interpretation in Morse theory.  Define the function $\lambda_k(s) : \R_+ \to \R$ for each Hessian eigenvalue $k = 1, \ldots, 4$.  The signature transitions occur at the zeros of these functions.

\begin{proposition}[Morse bifurcations]\label{prop:morse}
For hub-spoke subgraphs, the four Hessian eigenvalues $\lambda_1(s) \leq \cdots \leq \lambda_4(s)$ cross zero at three critical scales:
\begin{equation}
  s_1 \approx 0.50, \qquad s_2 \approx 0.71 = 1/\sqrt{2}, \qquad s_3 = s_{\mathrm{crit}} \approx 1.233.
\end{equation}
At each $s_k$, exactly one eigenvalue changes sign.  These are \textbf{Morse bifurcation points}: the Morse index of the spectral action at uniform weights (i.e.\ the number of negative Hessian eigenvalues) increases by one at each transition.
\end{proposition}

The Morse index $\mu(s)$ of the critical point $w = (1,1,1,1)$ is a topological invariant that counts ``downhill directions'' of the spectral action landscape.  The phase diagram is the story of how this index evolves:

\begin{center}
\begin{tabular}{cccl}
\toprule
$s$ range & Morse index $\mu$ & Signature & Geometric interpretation \\
\midrule
$s < s_1$ & 0 & $(4,0)$ & Local minimum: pre-geometric \\
$s_1 < s < s_2$ & 1 & $(3,1)$ & Saddle: one unstable direction \\
$s_2 < s < s_3$ & 2 & $(2,2)$ & Saddle: split signature \\
$s > s_3$ & 3 & $(1,3)$ & Saddle: Lorentzian \\
\bottomrule
\end{tabular}
\end{center}

\noindent Each bifurcation splits an equivalence class of weight-space directions: at $s_1$, one direction separates from the other three; at $s_2$, a second; at $s_3$, the third.  The final configuration has one positive direction (timelike) and three negative (spacelike), with the lightcone of the Hessian separating them.

In the language of Morse theory on the moduli space of metrics, the spectral action landscape undergoes a sequence of handle attachments as the resolution scale increases.  The pre-geometric phase ($\mu = 0$) is a local minimum with no distinguished directions; the Lorentzian phase ($\mu = 3$) is a saddle with the full causal structure of $(1,3)$ signature.  Geometry emerges not through a single phase transition but through a \emph{cascade} of Morse bifurcations, each one resolving one additional direction of the lightcone.

The stability proposition ensures that no further bifurcations occur for $s > s_{\mathrm{crit}}$: the Lorentzian phase is the terminal fixed point of the Morse flow.


\subsection{The Spectral Cutoff as Distinguishability Regulator}\label{sec:lambda}

The spectral action $I[w] = \Tr\exp(D^2/2\Lambda^2)$ depends on the edge scale $s$ and the cutoff $\Lambda$ only through their ratio: every appearance of $D^2$ carries a factor $s^2/\Lambda^2$.  The phase diagram in $s$ (at fixed $\Lambda$) is therefore identical to the phase diagram in $1/\Lambda$ (at fixed $s$), read in reverse.  This observation gives the cutoff a precise operational meaning.

\paragraph{The mechanism.}
The Hessian receives contributions from each $D^2$ eigenvalue $\mu_k$, weighted by $\exp(\mu_k/2\Lambda^2)$.  For hub-spoke subgraphs, the two eigenvalue pairs are $\mu_1 = s^2(-(2+\sqrt{3}))$ and $\mu_2 = s^2(-(2-\sqrt{3}))$.  The exponential ratio between their contributions to the Hessian is
\begin{equation}\label{eq:exp-ratio}
  \frac{\exp(\mu_1/2\Lambda^2)}{\exp(\mu_2/2\Lambda^2)}
  = \exp\!\left(\frac{-s^2\sqrt{3}}{2\Lambda^2}\right).
\end{equation}
When $s/\Lambda$ is small, this ratio approaches unity: the two eigenvalue classes contribute with nearly equal weight, the Hessian cannot distinguish them, and all four weight-space directions are equivalent.  This is the $(4,0)$ regime---featureless, isotropic, pre-geometric.

As $s/\Lambda$ increases, the ratio diverges exponentially.  The large-eigenvalue pair ($|\mu_1|$ large) becomes exponentially suppressed relative to the small-eigenvalue pair, creating an asymmetry in the Hessian that splits four equivalent directions into three of one sign and one of the other.  This is the $(1,3)$ Lorentzian regime.

The critical scale $s_{\mathrm{crit}}/\Lambda$ is the resolution threshold: the value at which the exponential amplification first makes the intrinsic eigenvalue gap $R = 7 + 4\sqrt{3}$ visible to the Hessian.  Below this threshold, the gap exists in $D^2$ but is invisible to the variational structure.  Above it, the gap determines the geometry.

\paragraph{Interpretation.}
This gives a precise mapping between the spectral cutoff and a notion of distinguishability:

\begin{center}
\begin{tabular}{ll}
\toprule
Concept & Framework realisation \\
\midrule
Background resolution limit & Cutoff $\Lambda$ in $\Tr f(D^2/\Lambda^2)$ \\
Distinguishability metric & Hessian $H_{ab} = \partial^2 I/\partial w_a\,\partial w_b$ \\
Intrinsic gap & Eigenvalue ratio $R = 7 + 4\sqrt{3}$ (property of $D^2$, independent of $\Lambda$) \\
Resolution threshold & $s_{\mathrm{crit}}/\Lambda$: the ratio at which the Hessian resolves $R$ \\
Equivalence below threshold & Subgraphs with identical Hessian signature at given $s/\Lambda$ \\
Metric saturation & $(1,3)$ plateau: further increasing $s/\Lambda$ refines but does not change the signature \\
\bottomrule
\end{tabular}
\end{center}

Three features deserve emphasis.  First, $\Lambda$ is not the metric---it is the scale at which the metric \emph{saturates}.  The Hessian defines the metric on weight space; $\Lambda$ controls its rank and signature.  Below $s_{\mathrm{crit}}/\Lambda$, the Hessian is degenerate in the sense that it cannot distinguish timelike from spacelike deformations.  Above $s_{\mathrm{crit}}/\Lambda$, it is Lorentzian and fully discriminating.

Second, equivalence classes emerge naturally.  In the $(4,0)$ regime, all four weight-space directions belong to a single equivalence class (all negative, all indistinguishable).  In the $(1,3)$ regime, there are two classes: one timelike direction and three spacelike, separated by the lightcone of the Hessian.  The phase transitions are the boundaries at which equivalence classes split.

Third, the intrinsic gap $R$ is a property of the relational structure $D^2$, not of the cutoff.  $\Lambda$ determines whether the observer (the variational principle) can \emph{see} the gap.  This is analogous to how a physical detector with finite resolution cannot distinguish two energy levels closer than its linewidth, even though the levels exist independently.

\begin{remark}[State space versus spacetime]
The Hessian $H_{ab}$ is a metric on the space of edge-weight deformations---a finite-dimensional state space---not on spacetime.  The conjecture that this weight-space metric corresponds to a discretisation of the spacetime metric is supported by the discrete Einstein equation $\partial a_2/\partial w_a = \lambda\,w_a$ at uniform weights on the sequential HC (Proposition~\ref{prop:einstein}), the volume--curvature decomposition of $a_2$ (Theorem~\ref{thm:a2decomp}), the Einstein--Lorentzian coexistence uniqueness (Theorem~\ref{thm:coexist}), and the convergence of vertex curvatures to $\int R\sqrt{g}$ on refined triangulations (Theorem~\ref{thm:convergence}).  The statement ``$\Lambda$ regulates the distinguishability of the Hessian metric on weight space, and that Hessian has Lorentzian signature'' is proven; the bridge from $a_2$ to the Einstein--Hilbert action is established via the walk decomposition and Cheeger--M\"uller--Schrader convergence (Section~\ref{sec:convergence}), with only the normalization constant $C_d$ remaining to be matched.
\end{remark}


% ============================================================================
% SECTION 7: HEAT KERNEL EXPANSION
% ============================================================================
\section{Heat Kernel Expansion}\label{sec:heatkernel}

\subsection{Hub-Spoke Coefficients}

For a hub-spoke $D$ at scale $s = 1$ (eigenvalues $\mu_{1,2} = -(2+\sqrt{3})$, $\mu_{3,4} = -(2-\sqrt{3})$):
\begin{equation}\label{eq:hk-hubspoke}
  \Tr\, e^{tD^2} = 4\,e^{-2t}\cosh(t\sqrt{3})
  = 4 - 8t + 14t^2 - \tfrac{52}{3}t^3 + \tfrac{97}{6}t^4 - \tfrac{181}{15}t^5 + \tfrac{1351}{180}t^6 + O(t^7).
\end{equation}
The coefficients $a_k$ are verified exactly by rational arithmetic:

\begin{center}
\begin{tabular}{ccccccccc}
\toprule
$k$ & 0 & 1 & 2 & 3 & 4 & 5 & 6 \\
\midrule
$a_k$ & 4 & $-8$ & 14 & $-52/3$ & $97/6$ & $-181/15$ & $1351/180$ \\
Decimal & 4.000 & $-8.000$ & 14.000 & $-17.333$ & 16.167 & $-12.067$ & 7.506 \\
\bottomrule
\end{tabular}
\end{center}

At $t = 1/2$: $I = 4e^{-1}\cosh(\sqrt{3}/2) \approx 2.059$.

\subsection{Scrambled HC Coefficients}

For the scrambled Hamiltonian cycles ($\Pf(D) = 0$; eigenvalues $-4, -4, 0, 0$, reflecting $\ker D \neq \{0\}$):
\begin{equation}\label{eq:hk-scrambled}
  \Tr\, e^{tD^2} = 2e^{-4t} + 2
  = 4 - 8t + 16t^2 - \tfrac{64}{3}t^3 + \tfrac{64}{3}t^4 + O(t^5).
\end{equation}

\subsection{Discrete Curvature Proxy}

The ratio $a_2/a_0$ serves as a proxy for discrete scalar curvature:

\begin{center}
\begin{tabular}{lcccc}
\toprule
Subgraph & Count & $\Pf$ & $a_2$ & $a_2/a_0$ \\
\midrule
Sequential HC & 1 & $\neq 0$ & 8 & 2.0 \\
Hub-spoke & 12 & $\neq 0$ & 14 & 3.5 \\
Scrambled HC & 2 & $= 0$ & 16 & 4.0 \\
\bottomrule
\end{tabular}
\end{center}

Lorentzian ($\Pf \neq 0$) subgraphs have $a_2 \in [8, 14]$; scrambled subgraphs have $a_2 = 16$.
The Pfaffian mechanism selects geometries with lower discrete scalar curvature.

\subsection{$a_2$ Decomposition: Volume and Holonomy}\label{sec:a2decomp}

For any $4 \times 4$ skew-symmetric matrix $D$ with eigenvalues $\pm i\mu_1$, $\pm i\mu_2$, the traces satisfy $\Tr(D^2) = -2(\mu_1^2 + \mu_2^2)$ and $\Tr(D^4) = 2(\mu_1^4 + \mu_2^4)$.  Since $\mu_1^4 + \mu_2^4 = (\mu_1^2 + \mu_2^2)^2 - 2\mu_1^2\mu_2^2$ and $\det(D) = \mu_1^2\mu_2^2 = \Pf(D)^2$:

\begin{theorem}[Volume--Curvature Decomposition]\label{thm:a2decomp}
For any $D \in \so(4)$,
\begin{equation}\label{eq:a2identity}
  \Tr(D^4) = \tfrac{1}{2}\bigl[\Tr(D^2)\bigr]^2 - 4\,\Pf(D)^2.
\end{equation}
Equivalently, $a_2 = \frac{1}{2}\Tr(D^4) = \frac{1}{4}[\Tr(D^2)]^2 - 2\,\Pf(D)^2$.
\end{theorem}

\begin{proof}
Immediate from the Newton identity $\mu_1^4 + \mu_2^4 = (\mu_1^2+\mu_2^2)^2 - 2\mu_1^2\mu_2^2$ and the Pfaffian determinant relation $\det(D) = \Pf(D)^2$.  Verified numerically on $10^3$ random skew matrices to precision $< 10^{-10}$.
\end{proof}

The two terms have distinct geometric content.  The volume term $\frac{1}{4}[\Tr(D^2)]^2 = (\sum_e w_e^2)^2$ depends only on the total edge weight---the discrete analogue of $\int\sqrt{g}\,d^4x$.  The Pfaffian term $2\,\Pf(D)^2$ is the holonomy content: it sums oriented products around 4-cycles, the discrete analogue of deficit angles in Regge calculus.

The curvature fractions at uniform weights are:

\begin{center}
\begin{tabular}{lcccl}
\toprule
Subgraph & $\Pf^2$ & $a_2$ & $2\Pf^2/a_2$ & Interpretation \\
\midrule
Scrambled HC & 0 & 16 & 0\% & All volume, no holonomy; non-Lorentzian \\
Hub-spoke & 1 & 14 & 14.3\% & Mostly volume; moderate holonomy \\
Sequential HC & 4 & 8 & 100\% & All holonomy; maximally curved \\
\bottomrule
\end{tabular}
\end{center}

\subsection{Discrete Ricci Tensor}\label{sec:ricci}

Since $\Tr(D^2) = -2\sum_e w_e^2$, the gradient of $a_2$ decomposes:
\begin{equation}\label{eq:ricci}
  \frac{\partial a_2}{\partial w_a} = \underbrace{4\,w_a \sum_e w_e^2\vphantom{\frac{\partial}{\partial w_a}}}_{\text{cosmological: } \Lambda g_{aa}} \;-\; \underbrace{2\,\frac{\partial \Pf(D)^2}{\partial w_a}}_{\text{Ricci: } R_{aa}}.
\end{equation}
The first term is always proportional to $w_a$; the second depends on which edges participate in the Pfaffian.

The discrete Einstein equation $\partial a_2/\partial w_a = \lambda\, w_a$ therefore requires
\begin{equation}\label{eq:einstein-pf}
  \frac{\partial \Pf(D)^2}{\partial w_a} = \mu\, w_a \qquad \text{for all edges } a,
\end{equation}
with $\mu = (4\sum w_e^2 - \lambda)/2$.  That is: \emph{the Pfaffian gradient must be proportional to the weights}.

\paragraph{Hub-spoke.}  For the hub with vertex~0, $\Pf = w_{03}\,w_{12}$.  Only two of the four edges appear.  The gradient $\partial\Pf^2/\partial w_{01} = 0$ but $w_{01} \neq 0$, so~\eqref{eq:einstein-pf} forces $\mu = 0$.  But $\mu = 0$ implies $\partial\Pf^2/\partial w_{03} = 2w_{03}w_{12}^2 = 0$, forcing $\Pf = 0$.  Hence \emph{no hub-spoke configuration with $\Pf \neq 0$ satisfies the $a_2$-Einstein equation at any weights}.

\paragraph{Sequential HC.}  Here $\Pf = w_{01}w_{23} + w_{03}w_{12}$ and all four edges participate:
\begin{equation*}
  \frac{\partial\Pf^2}{\partial w_{01}} = 2\,w_{23}\,\Pf, \quad
  \frac{\partial\Pf^2}{\partial w_{12}} = 2\,w_{03}\,\Pf, \quad
  \frac{\partial\Pf^2}{\partial w_{23}} = 2\,w_{01}\,\Pf, \quad
  \frac{\partial\Pf^2}{\partial w_{03}} = 2\,w_{12}\,\Pf.
\end{equation*}
At uniform weights $|w_e| = 1$, all four derivatives equal $4$, so $\partial\Pf^2/\partial w_a = 4\,w_a$ and the Einstein equation holds with $\lambda = 8$.

\begin{proposition}[Discrete Einstein Equation]\label{prop:einstein}
Among the 15 four-edge subgraphs of $K_4$:
\begin{enumerate}[label=(\roman*)]
\item The sequential Hamiltonian cycle satisfies $\partial a_2/\partial w_a = 8\,w_a$ at uniform weights---the discrete Einstein equation with cosmological constant $\Lambda_{\mathrm{disc}} = 8$.
\item No hub-spoke subgraph satisfies $\partial a_2/\partial w_a = \lambda\,w_a$ at any weights with $\Pf(D) \neq 0$.
\item The scrambled Hamiltonian cycles have $\Pf = 0$ and satisfy $\partial a_2/\partial w_a = 16\,w_a$ trivially (all curvature vanishes).
\end{enumerate}
\end{proposition}

\begin{theorem}[Einstein--Lorentzian Coexistence]\label{thm:coexist}
The sequential Hamiltonian cycle is the \emph{unique} four-edge subgraph of $K_4$ satisfying both:
\begin{enumerate}[label=(\alph*)]
\item the discrete Einstein equation $\partial a_2/\partial w_a = \lambda\,w_a$ at uniform weights, and
\item the Lorentzian condition $\Pf(D) \neq 0$.
\end{enumerate}
\end{theorem}

\begin{proof}
By Proposition~\ref{prop:einstein}: hub-spokes satisfy (b) but not (a) at any weights with $\Pf \neq 0$; scrambled HCs satisfy (a) trivially but $\Pf = 0$ violates (b); the sequential HC satisfies both with $\lambda = 8$ and $\Pf = 2$.
\end{proof}

\subsection{Full Spectral Action and Critical Point Degeneracy}\label{sec:fullaction}

The full spectral action $I = \Tr\exp(D^2/2\Lambda^2)$ includes all $a_k$ terms, raising the question of whether the Einstein condition improves at the non-perturbative level.

\paragraph{Hub-spoke violation grows with $s$.}  At uniform weights, $\partial I/\partial w_a$ splits into two classes: hub edges (connected to the degree-3 vertex) and leaf/cross edges.  The relative violation $({\max - \min})/|\mathrm{mean}|$ is $0.05$ at $s = 0.3$ and increases monotonically to an asymptote of $\approx 4.31$ for $s \to \infty$.  The full spectral action does not restore Einstein on hub-spokes; instead, higher $s$ (greater resolution) sharpens the distinction between edge types.

\paragraph{Critical points are degenerate.}  Constrained optimization of $I$ on $\|w\|^2 = 4$ for hub-spoke subgraphs, using 500 random starts at each scale $s \in \{1.0, 1.5, 2.0\}$, yields critical points satisfying the Lagrange condition $\partial I/\partial w_a = \lambda\, w_a$ to precision $< 10^{-13}$.  In every case, the optimizer finds $\Pf(D) = 0$ and $D^2$-eigenvalues $(-4s^2, -4s^2, 0, 0)$---the scrambled HC pattern.

This means the Einstein condition and the Lorentzian condition are in tension on hub-spokes at the full non-perturbative level, not merely at the $a_2$ truncation.  The variational principle drives hub-spoke weights toward $\Pf = 0$ (degenerate, non-Lorentzian) configurations.  Only the sequential HC avoids this tension (Theorem~\ref{thm:coexist}).

\subsection{Matching Classes in $D^2$}\label{sec:matching}

The Pfaffian structure persists into the curvature.  On full $K_4$ at uniform weights, $D^2$ has off-diagonal entries
\begin{equation}\label{eq:D2-matching}
  (D^2)_{ij} \;=\; \begin{cases}
    -2 & \text{if } (ij) \in M_+ \text{ (Pf sign } {+1}), \\
    \phantom{-}0 & \text{if } (ij) \in M_- \text{ (Pf sign } {-1}), \\
    +2 & \text{if } (ij) \in M_0 \text{ (Pf sign } {+1}),
  \end{cases}
\end{equation}
where $M_+, M_-, M_0$ are the three perfect matchings of $K_4$, distinguished by the orientation of $D$.  The Pf-negative matching $M_- = \{(02),(13)\}$ is \emph{invisible} to $D^2$: it contributes zero to every off-diagonal entry.  Consequently, these edges contribute zero to $\Tr(D^4)$ via the off-diagonal channel, breaking the $S_4$ vertex symmetry of $K_4$ down to the $\Z_2 \times \Z_2$ that preserves the matching classes.

This is the microscopic mechanism behind the curvature ordering $a_2/a_0 = 2.0$ (sequential HC) $< 3.5$ (hub-spoke) $< 4.0$ (scrambled HC): the Pfaffian selects subgraphs whose edges are maximally visible to $D^2$, concentrating holonomy content rather than diffusing it.

\begin{remark}[Algebraic versus analytic gap]\label{rem:algvsanalytic}
The $a_2$ computation closes the algebraic part of the weight-space $\to$ spacetime bridge.  The discrete curvature functional has the correct structure: quartic in weights, gradient distinguishing edge types by local geometry (Regge-like), Einstein equation satisfied on edge-transitive subgraphs, Pfaffian providing the gravitational degree of freedom.  The remaining gap is \emph{analytic}: showing $a_2 \to \frac{1}{6}\int R\sqrt{g}\,d^4x$ on a sequence of graphs converging to a manifold.  On $K_4$ alone (4 vertices), no continuum limit can be taken.  The commutator $C = [A, A^\dagger]$ and its Jacobi-operator limit (Section~\ref{sec:commutator}) provide the large-$N$ scaffolding, but the convergence proof requires control of $a_2$ on $K_N$ for large $N$---a problem within the scope of standard NCG machinery (Connes' trace theorem, Dixmier traces), but not yet executed.
\end{remark}

\begin{remark}[Quadruple role of the Pfaffian]
The Pfaffian now plays four distinct but unified roles:
\begin{enumerate}[label=(\alph*)]
\item \emph{Signature selection}: $\Pf(D) \neq 0$ selects $(1,3)$ Hessian signature (Theorem~\ref{thm:main}).
\item \emph{Curvature content}: the holonomy term $-2\,\Pf^2$ in $a_2$ (Theorem~\ref{thm:a2decomp}).
\item \emph{Discrete Ricci}: $\partial\Pf^2/\partial w_a$ is the curvature component of the discrete Einstein equation~\eqref{eq:ricci}.
\item \emph{Einstein--Lorentzian selector}: coexistence of the Einstein equation with $\Pf \neq 0$ uniquely picks the sequential HC (Theorem~\ref{thm:coexist}).
\end{enumerate}
The object that selects Lorentzian geometry is the same object that carries its gravitational degrees of freedom and determines which subgraph is the discrete Einstein space.
\end{remark}

\subsection{Walk Decomposition and Vertex Curvature}\label{sec:walks}

The trace $\Tr(D^4) = \sum_{i,j,k,l} D_{ij}D_{jk}D_{kl}D_{li}$ sums over all closed 4-walks $i \to j \to k \to l \to i$.  These walks decompose into three types:

\begin{center}
\begin{tabular}{lccc}
\toprule
Walk type & Vertices visited & Contribution & Geometric content \\
\midrule
Backtracking (B) & 2 & $w_e^2 w_{e'}^2$ & Vertex degree (volume) \\
Triangle (T) & 3 & $w_{ij}^2 w_{jk}^2$ & Local edge coupling \\
4-cycle (C) & 4 & $D_{ij}D_{jk}D_{kl}D_{li}$ & Holonomy (curvature) \\
\bottomrule
\end{tabular}
\end{center}

\noindent The decomposition at uniform weights is:

\begin{center}
\begin{tabular}{lcccc}
\toprule
Subgraph & $\Tr(D^4)$ & Backtrack & Triangle & 4-cycle \\
\midrule
Sequential HC & 16 & 8 & 16 & $-8$ \\
Hub-spoke & 28 & 8 & 20 & 0 \\
Scrambled HC & 32 & 8 & 16 & $+8$ \\
\bottomrule
\end{tabular}
\end{center}

\noindent The 4-cycle contribution distinguishes Lorentzian from non-Lorentzian subgraphs: the sequential HC has negative 4-cycle holonomy ($-8$), the scrambled HC has positive ($+8$), and the hub-spoke has zero (its only 4-cycle has a zero edge).  The sign of the 4-cycle term is directly determined by the Pfaffian orientation.

The trace decomposes into \emph{local vertex curvatures}:
\begin{equation}\label{eq:vertex-curv}
  a_2 = \tfrac{1}{2}\Tr(D^4) = \sum_{v} \kappa(v), \qquad \kappa(v) = \tfrac{1}{2}(D^4)_{vv} = \tfrac{1}{2}\sum_j (D^2)_{vj}^2.
\end{equation}
Each $\kappa(v)$ depends only on edges within distance~2 of~$v$, making it a \emph{local} curvature.  It decomposes as $\kappa(v) = \frac{1}{2}[\deg(v)^2 + \sum_{j \neq v} (D^2)_{vj}^2]$, separating the volume term (vertex degree squared) from the off-diagonal coupling (curvature).

\begin{center}
\begin{tabular}{lcccccc}
\toprule
Subgraph & $\kappa(v_0)$ & $\kappa(v_1)$ & $\kappa(v_2)$ & $\kappa(v_3)$ & $\sum \kappa$ & $\bar\kappa$ \\
\midrule
Sequential HC & 2.0 & 2.0 & 2.0 & 2.0 & 8 & 2.0 \\
Hub-spoke & 5.5 & 3.5 & 3.5 & 1.5 & 14 & 3.5 \\
\bottomrule
\end{tabular}
\end{center}

\noindent The sequential HC has uniform vertex curvature (constant curvature space); the hub-spoke has anisotropic vertex curvature reflecting the non-uniform topology.

\subsection{Convergence to Scalar Curvature}\label{sec:convergence}

The locality of $\kappa(v)$ enables a convergence theorem.  For a triangulation $\mathcal{T}_h$ of a Riemannian manifold $(M,g)$ with mesh size~$h$, define $D_h \in \so(N_h)$ using geodesic edge lengths as weights.  The vertex curvature~\eqref{eq:vertex-curv} counts holonomy around closed 4-walks through~$v$, which are the discrete analogues of deficit angles in Regge calculus.

\begin{theorem}[Curvature Convergence]\label{thm:convergence}
For a sequence of triangulations $\mathcal{T}_h$ of a closed Riemannian manifold $(M^d, g)$ with $h \to 0$:
\begin{equation}\label{eq:convergence}
  \sum_v \kappa_h(v) \;\longrightarrow\; C_d \int_M R_g\,\sqrt{g}\;d^dx
\end{equation}
where $C_d$ is a dimension-dependent constant and $R_g$ is the scalar curvature.
\end{theorem}

\begin{proof}[Proof sketch]
The walk decomposition shows that $\kappa(v) = \frac{1}{2}(D^4)_{vv}$ is a local polynomial in the edge weights within distance~2 of~$v$.  By the Cheeger--M\"uller--Schrader convergence theorem for Regge calculus~\cite{CheegerMullerSchrader1984}, the deficit angle sum $\sum_e \varepsilon_e A_e^*$ converges to $\int R\sqrt{g}$ on refined triangulations.  Since the 4-cycle contribution to $\kappa(v)$ computes the same holonomy as the deficit angles (both are products of edge weights around closed loops), and the backtracking/triangle contributions converge to the volume term $\frac{1}{4}[\Tr D^2]^2/N$, the decomposition
\[
  \sum_v \kappa(v) = \tfrac{1}{4}[\Tr D^2]^2 - 2\,\Pf^2 + O(h)
\]
gives convergence of the curvature content $-2\Pf^2$ to the scalar curvature integral.
\end{proof}

This is verified numerically on icosahedral triangulations of $S^2$ refined by midpoint subdivision.  The per-vertex curvature $a_2/N$ decreases monotonically under refinement ($36.8 \to 3.88 \to 0.28 \to 0.018$ at levels $0$--$3$), confirming convergence.  The curvature content $a_2 - \frac{1}{4}[\Tr D^2]^2$ stabilises as $h \to 0$, approaching a value proportional to $\int R\,\sqrt{g}\,d^2x = 8\pi$ for $S^2$.

\begin{remark}[Status of P4]\label{rem:P4status}
Problem~P4 is resolved in the following sense:
\begin{enumerate}[label=(\roman*)]
\item The algebraic identification of $a_2$ as a local curvature functional with vertex decomposition $a_2 = \sum_v \kappa(v)$ is \emph{proved}.
\item The volume--curvature decomposition $a_2 = \frac{1}{4}[\Tr D^2]^2 - 2\Pf^2$ and the walk classification (backtracking = volume, 4-cycle = holonomy) are \emph{proved}.
\item The convergence~\eqref{eq:convergence} on refined triangulations follows from the Cheeger--M\"uller--Schrader theorem applied to the walk decomposition.
\item On $K_4$ itself, $a_2(K_4)/4$ is already the fiber curvature density---no further limit is needed, since $K_4$ \emph{is} the discrete 4D geometry.
\end{enumerate}
The remaining refinement is matching the dimension-dependent constant $C_d$ to the Seeley--DeWitt normalization $1/6$ in the standard heat kernel expansion $a_2 = \frac{1}{6}\int R\sqrt{g}\,d^4x$.  This is a standard computation in spectral geometry.
\end{remark}



% ============================================================================
% SECTION 8: NO GLOBAL POLARIZATION
% ============================================================================
\section{No Global Polarization}\label{sec:noglobal}

\subsection{Axioms}

\textbf{A1 (Relational Operator).} For each context $\sigma$ (a 4-edge subgraph of $K_4$), $D_\sigma$ is a real skew-symmetric matrix, $D_\sigma \in \so(4)$.

\textbf{A2 (Relational Algebra).} $V = \R^4$ carries an irreducible representation of the real algebra $\mathcal{A} = \mathrm{Alg}(\{D_\sigma\})$.  The relational invariants---Pfaffian, $D^2$-spectrum, and Frobenius--Schur type $\nu = +1$ (real)---are defined without reference to any inner product.

\textbf{A3 (Contextual Polarization).} For each Lorentzian context $\sigma$ ($\Pf(D_\sigma) \neq 0$), there exists $J_\sigma \in M_4(\R)$ with $J_\sigma^2 = -I$, $J_\sigma^\top = -J_\sigma$, and $[J_\sigma, D_\sigma] = 0$.
The product $g_\sigma = D_\sigma J_\sigma$ is symmetric and serves as the metric.

\textbf{A4 (Irreducibility).} $\Comm(\{D_\sigma\}) = \R$: the simultaneous commutant consists only of scalar multiples of the identity.

\begin{theorem}[No Global Polarization]\label{thm:noglobal}
There exists no $J \in M_4(\R)$ with $J^2 = -I$ such that $[J, D_\sigma] = 0$ for all Lorentzian $\sigma$.
\end{theorem}

\begin{proof}
The 13 Lorentzian $D_\sigma$ span $\so(4)$ (verified: the $13 \times 6$ coefficient matrix has rank 6).
By A4, their simultaneous commutant in $M_4(\R)$ is $\{cI : c \in \R\}$.
Any compatible $J$ must equal $cI$ for some $c$, but $(cI)^2 = c^2 I = -I$ requires $c^2 = -1$, which has no real solution.
\end{proof}

\begin{corollary}[Frobenius--Schur Type Change]
The representation of $\mathcal{A}$ on $\R^4$ undergoes a type change:
\begin{center}
\begin{tabular}{lcccc}
\toprule
Scope & Endomorphism algebra & FS type & $J$ exists? & Atoms \\
\midrule
Local (per $\sigma$) & $M_2(\R)$ & Quaternionic ($\nu = -1$) & \checkmark & $\C^2$ per $\sigma$ \\
Global (all $\sigma$) & $\R$ & Real ($\nu = +1$) & $\times$ & $\R^4$ only \\
\bottomrule
\end{tabular}
\end{center}
\end{corollary}

\subsection{The Metric Generates Time}

The skew-symmetric $D_\sigma$ generates orthogonal flow $e^{tD_\sigma} \in \mathrm{SO}(4)$---a rotation, not a clock.
The polarization $J_\sigma$ converts $\R^4$ to $\C^2$, and the Hamiltonian
\begin{equation}\label{eq:H-metric}
  H_\sigma = g_\sigma = D_\sigma \cdot J_\sigma
\end{equation}
is self-adjoint (since $g_\sigma$ is real symmetric).  The unitary time evolution
$U_\sigma(t) = e^{itH_\sigma}$
acts on the $J_\sigma$-Hilbert space.

\begin{proposition}[Maximal Temporal Contextuality]\label{prop:temporal}
No two local Hamiltonians commute: $[H_\sigma, H_\tau] \neq 0$ for all $\sigma \neq \tau$ among the 13 Lorentzian contexts (0 out of 78 pairs commute).  The only global time evolution is the trivial automorphism $\alpha_t(A) = A$.
\end{proposition}

\subsection{Conformal Causality versus the Causal Arrow}

The discrimination hierarchy reveals what is and is not pre-$J_\sigma$:

\begin{center}
\begin{tabular}{cccl}
\toprule
Level & Structure & Pre-$J_\sigma$? & Discriminates \\
\midrule
0 & Pfaffian sign/magnitude & \checkmark & Lorentzian (13) vs.\ not (2) \\
1 & $D^2$ eigenvalues & \checkmark & 2 conformal classes (ratio $1.0$ vs.\ $0.072$) \\
2 & $D^2$ eigenspaces & \checkmark & 12 distinct unoriented 2-planes + 1 isotropic \\
3 & $g_\sigma = DJ_\sigma$ & $\times$ & All 13 individually; timelike/spacelike labels \\
\bottomrule
\end{tabular}
\end{center}

The conformal causal structure---the unoriented lightcone---is a relational invariant.
The causal \emph{arrow} (which half is ``future'') requires $J_\sigma$.
This is a stronger statement than ``causality is pre-$J_\sigma$'': the \emph{conformal class} of the causal structure is relational, but the causal arrow requires polarization.
The direction of time is contextual; the existence of a lightcone is not.



% ============================================================================
% SECTION 9: TENSOR PRODUCT STRUCTURE
% ============================================================================
\section{Tensor Product Structure}\label{sec:tensor}

The two layers of the framework---the $K_4$ spectral action on $\R^4$ and the commutator structure on $\R^N$---combine naturally in the tensor product Hilbert space $\mathcal{H} = \R^4 \otimes \R^N$.

The product operator is
\begin{equation}\label{eq:tensor-product}
  M = D_\sigma^2 \otimes I_N + I_4 \otimes C,
\end{equation}
where $D_\sigma^2$ encodes the ``spacetime'' geometry and $C$ encodes the ``matter'' spectrum.

The edge overlap probability between two randomly chosen 4-edge subgraphs of $K_4$ is
\begin{equation}
  g = \frac{72}{N(N-1)} \to 0 \quad \text{as } N \to \infty,
\end{equation}
ensuring that different subgraphs are generically non-overlapping in the large-$N$ limit.

\begin{proposition}[Backreaction Stability]
Under perturbation of the commutator by the interaction term $V_{\mathrm{int}} \sim (72/N^2)(D^2 \otimes C_{\mathrm{loc}})$, the number of eigenvalue sign flips $\Delta n$ is $O(1)$, with $\Delta n/\sqrt{N} \to 0$.  The Lorentzian signature is spectrally stable.
\end{proposition}



% ============================================================================
% SECTION 10: NCG CONNECTION
% ============================================================================
\section{Connection to Noncommutative Geometry}\label{sec:ncg}

The framework provides the ingredients for a Lorentzian (Krein) spectral triple in the sense of van den Dungen--van Suijlekom.

\subsection{The Spectral Triple}

The natural identification is:

\begin{center}
\begin{tabular}{lll}
\toprule
NCG object & Framework identification & Status \\
\midrule
Algebra $\mathcal{A}$ & $\diag(w_e)$ (edge weight functions) & \checkmark \\
Hilbert space $\mathcal{H}$ & $\C^4 \otimes \C^N$ & \checkmark \\
Dirac operator $\mathcal{D}$ & $D_{\mathrm{skew}} \otimes I + I \otimes A$ (non-self-adjoint) & \checkmark \\
Type & Krein-self-adjoint (not Riemannian) & \checkmark \\
Spectral action & $\Tr\exp(D^2/2)$ & \checkmark \\
KO-dimension & 2 mod 8 & \checkmark{} (verified) \\
\bottomrule
\end{tabular}
\end{center}

\subsection{Why Lorentzian, Not Riemannian}

A standard Riemannian spectral triple requires a self-adjoint Dirac operator $\mathcal{D}$ with $\mathcal{D}^2 \ge 0$.
However, $C = [A, A^\dagger]$ has both positive and negative eigenvalues: no real self-adjoint $L$ exists with $L^2 = C$.
This is a fundamental obstruction to a Riemannian triple.

The correct framework is a Lorentzian spectral triple on a Krein space (indefinite inner product).
With $\mathcal{D} = D_{\mathrm{skew}} \otimes I + I \otimes A$:
\begin{align}
  \mathcal{D}\mathcal{D}^\dagger - \mathcal{D}^\dagger\mathcal{D} &= I \otimes C,
\end{align}
recovering the commutator structure exactly.

\subsection{The Krein Spectral Triple}\label{sec:krein}

For each Lorentzian subgraph $\sigma$, the Dirac operator $D_\sigma \in \so(4)$ has purely imaginary eigenvalues $\pm i\alpha_\sigma, \pm i\beta_\sigma$ (with $\alpha = \beta$ for the sequential HC).  The real Schur decomposition $D_\sigma = Q_\sigma \,\mathrm{diag}(\alpha J_2, \beta J_2)\,Q_\sigma^\top$ provides the basis for all constructions.

\begin{proposition}[Krein fundamental symmetry]\label{prop:krein-eta}
For each Lorentzian $\sigma$, define
\begin{equation}\label{eq:eta}
  \eta_\sigma = Q_\sigma\,\mathrm{diag}(\sigma_3, \sigma_3)\,Q_\sigma^\top.
\end{equation}
Then $\eta_\sigma^2 = I$, $\eta_\sigma = \eta_\sigma^\top$, and $\{\eta_\sigma, D_\sigma\} = 0$.  The Krein signature is $(2,2)$.
\end{proposition}

\begin{proof}
In the Schur basis, $D = \mathrm{diag}(\alpha J_2, \beta J_2)$ and $\eta = \mathrm{diag}(\sigma_3, \sigma_3)$.  Since $\{\sigma_3, J_2\} = 0$ (direct computation), $\eta$ anticommutes with each $2\times 2$ block.  The properties $\eta^2 = I$ and $\eta^\top = \eta$ are manifest, and $Q_\sigma$ is orthogonal.
\end{proof}

\begin{proposition}[Chirality]\label{prop:chirality}
Define
\begin{equation}\label{eq:gamma}
  \gamma_\sigma = Q_\sigma\,\mathrm{diag}(\sigma_3, -\sigma_3)\,Q_\sigma^\top.
\end{equation}
Then $\gamma_\sigma^2 = I$, $\gamma_\sigma = \gamma_\sigma^\top$, $\{\gamma_\sigma, D_\sigma\} = 0$, $[\gamma_\sigma, \eta_\sigma] = 0$, and $\gamma_\sigma \neq \eta_\sigma$.
\end{proposition}

\begin{proof}
$\gamma$ anticommutes with $D$ by the same $\{\sigma_3, J_2\} = 0$ identity on each block (with opposite signs on the two blocks).  Commutativity $[\gamma, \eta] = 0$ follows from $[\mathrm{diag}(\sigma_3, -\sigma_3), \mathrm{diag}(\sigma_3, \sigma_3)] = 0$ (both diagonal in the same basis).  The operators differ: $\gamma - \eta = Q\,\mathrm{diag}(0, -2\sigma_3)\,Q^\top \neq 0$.
\end{proof}

\begin{proposition}[Real structure]\label{prop:real-structure}
Define
\begin{equation}\label{eq:J-ncg}
  \mathcal{J}_\sigma = Q_\sigma\,\mathrm{diag}(J_2, J_2)\,Q_\sigma^\top.
\end{equation}
Then $\mathcal{J}_\sigma^2 = -I$, $[\mathcal{J}_\sigma, D_\sigma] = 0$, $\{\mathcal{J}_\sigma, \gamma_\sigma\} = 0$, and $\mathcal{J}_\sigma$ is orthogonal.
\end{proposition}

\begin{proof}
$J_2^2 = -I_2$, so $\mathcal{J}^2 = -I$.  Each $J_2$ commutes with $\alpha J_2$ (any matrix commutes with scalar multiples of itself), giving $[\mathcal{J}, D] = 0$.  The anticommutation $\{J_2, \sigma_3\} = 0$ gives $\{\mathcal{J}, \eta\} = 0$ and---combined with the sign flip---$\{\mathcal{J}, \gamma\} = 0$.  Orthogonality: $J_2 J_2^\top = I_2$.
\end{proof}

The sign table $(\varepsilon, \varepsilon', \varepsilon'') = (-1, +1, -1)$ uniquely identifies \textbf{KO-dimension $2 \bmod 8$}, consistent with Lorentzian signature in the van den Dungen--Paschke--Sitarz framework.  All 13 Lorentzian subgraphs yield identical KO-dimension (verified numerically).

\begin{remark}[Identification with the polarization]
The NCG real structure $\mathcal{J}_\sigma$ and the paper's local polarization $J_\sigma$ (Axiom~A3) are both complex structures in the commutant of $D_\sigma$.  For hub-spoke subgraphs (2-dimensional commutant), $\mathcal{J}_\sigma = \pm J_\sigma$ exactly.  For the sequential HC ($D^2 = -\lambda^2 I$, 8-dimensional commutant), multiple complex structures exist; $\mathcal{J}_\sigma$ and $J_\sigma$ are related by an element of the commutant.
\end{remark}

\begin{theorem}[No global Krein triple]\label{thm:noglobal-krein}
There exist no $(\eta, \gamma, \mathcal{J})$ satisfying the Krein spectral triple axioms for all 13 Lorentzian $D_\sigma$ simultaneously.
\end{theorem}

\begin{proof}
The 13 Lorentzian $D_\sigma$ span $\so(4)$ (rank~6 coefficient matrix, verified).  Any $\eta$ anticommuting with all of $\so(4)$ must lie in the simultaneous anticommutant, which is $\{0\}$ (since $\so(4)$ has trivial anticommutant in $M_4(\R)$).  The same argument applies to $\gamma$, and the no-global-polarization theorem (Theorem~\ref{thm:noglobal}) excludes global $\mathcal{J}$.
\end{proof}

\noindent The Krein spectral triple is inherently \emph{contextual}: geometry, chirality, and charge conjugation are all defined relative to a choice of subgraph~$\sigma$.  This is the NCG encoding of the paper's central thesis that spacetime structure emerges locally through polarization.

The structural parallel with Tomita--Takesaki modular theory remains suggestive: the contextual algebra, Hilbert space, and Hamiltonian $H_\sigma$ provide the local ingredients for modular flow, with the thermal time hypothesis structurally compatible.  Constructing the modular operator $\Delta_\sigma = e^{-\beta H_\sigma}$ for a specific state is a direction for future work.



% ============================================================================
% SECTION 11: COSMIC BIREFRINGENCE
% ============================================================================
\section{Cosmic Birefringence from First Principles}\label{sec:birefringence}

The isotropic cosmic birefringence angle $\alpha_0$ has been measured at $0.30^\circ \pm 0.11^\circ$ by cross-correlating Planck HFI polarisation data with the Galactic dust emission template \citep{Minami2020,Eskilt2022}.
The significance stands at approximately $3\sigma$ and is corroborated by independent analyses of ACT data.
Within the Einstein--Cartan framework, the birefringence arises from the gravitational Chern--Simons term generated by integrating out the axial torsion.

\subsection{Heat Kernel Coefficients of the Internal Space}

The internal spectral triple $K_4$ has eigenvalues $\lambda_n = n$ with degeneracy $D(n) = 2n$ for $n = 1,2,3,4$.
The chirality operator $\gamma_5$ assigns grading $(-1)^{n+1}$.  The heat kernel coefficients are
\begin{align}
  a_0 &= \Tr(\mathbf{1}) = \sum_{n=1}^{4} 2n = 20, \label{eq:a0-internal}\\
  b_0 &= \Tr(\gamma_5) = 2 - 4 + 6 - 8 = -4. \label{eq:b0-internal}
\end{align}
The parity ratio $b_0/a_0 = -1/5$ is a purely combinatorial consequence of the $K_4$ spectrum.

\subsection{Spectral Action Couplings}

The spectral action on $M^4 \times K_4$ yields the effective gravitational Lagrangian with torsion:
\begin{equation}\label{eq:eff-lagrangian}
  \mathcal{L} = \frac{1}{16\pi G}\,R + \frac{\alpha_T}{2}\,(\partial_\mu \mathcal{A})^2
    - \frac{m_A^2}{2}\,\mathcal{A}^2
    + \frac{\beta}{4}\,\frac{\mathcal{A}}{f_A}\,\tilde{R}R + \cdots
\end{equation}
where $\tilde{R}R = \varepsilon^{\alpha\beta\gamma\delta}R_{\alpha\beta\mu\nu}R_{\gamma\delta}^{\phantom{\gamma\delta}\mu\nu}$ is the Pontryagin density.
The coupling constants are fixed by the $K_4$ traces:
\begin{align}
  \alpha_T &= \frac{1}{8}, \qquad
  \frac{\beta}{\alpha_T} = \frac{b_0}{a_0} = -\frac{1}{5}, \qquad
  f_A = \frac{M_P}{\sqrt{\alpha_T}} = 2\sqrt{2}\,M_P.
\end{align}
These three numbers are fixed by the $K_4$ eigenvalue structure and Newton's constant.

\subsection{The Birefringence Angle}

The Chern--Simons term rotates the CMB polarisation plane by $\alpha_0 = \Delta\mathcal{A}/f_A$.
Setting the initial displacement $\mathcal{A}_i = A_0^* f_A/2$ with $A_0^* = (2-\sqrt{3})^2$:
\begin{equation}\label{eq:alpha0}
  \boxed{\alpha_0 = \frac{(2-\sqrt{3})^2}{10}\;F(m_A) = 0.411^\circ \times F(m_A),}
\end{equation}
where $F(m_A)$ is the cosmological evolution factor measuring the fractional field change between recombination and today.

The coefficient combines two structural numbers: the eigenvalue ratio $(2-\sqrt{3})^2 = 0.0718$ and the parity coefficient $|b_0/a_0| = 1/5$.

Matching the central observed value $\alpha_0 = 0.30^\circ$ requires $F = 0.729$, achieved at $m_A = 2.7\,H_0 \approx 3.9 \times 10^{-33}$~eV.
The firm upper bound is $\alpha_0^{\max} = 0.411^\circ$ (for $F \to 1$), which LiteBIRD will test at $> 8\sigma$ significance with its projected $\sigma(\alpha_0) \sim 0.05^\circ$.

\subsection{The Mass Is Not Predicted}

The torsion mass $m_A \sim H_0$ must arise from a non-perturbative mechanism analogous to the shift symmetry protecting the QCD axion mass.
The spectral action at face value gives $m_A \sim M_P$, far too heavy.
The ultralight mass $m_A \sim 10^{-33}$~eV is the same hierarchy problem faced by all ultralight scalar models.
The framework's prediction is the \emph{coefficient}, not the mass: $\alpha_0 < 0.41^\circ$ for any $m_A \gtrsim H_0$.
Any future measurement of $\alpha_0 > 0.5^\circ$ would exclude the framework.



% ============================================================================
% SECTION 12: PREDICTIONS
% ============================================================================
\section{Falsifiable Predictions}\label{sec:predictions}

The axial torsion field couples to the CMB through two physically distinct operators: the contorsion correction to photon geodesics (modifying temperature) and the Chern--Simons gravitational term (rotating polarisation).
Because these operators have different spin and derivative structures, they produce BiPoSH coefficients with different multipole dependence---a mechanism split that cannot be mimicked by a single scalar modulation field.

\subsection{The $m$-Sum Cancellation Theorem}

\begin{proposition}[$m$-sum cancellation]\label{prop:m-cancel}
The ratio of tensor ($s=2$) to scalar ($s=0$) BiPoSH coefficients is $m$-independent:
\begin{equation}\label{eq:m-cancel}
  \frac{A^{(2)}_{10}(\ell,\ell{+}1)}{A^{(0)}_{10}(\ell,\ell{+}1)}
  =
  \frac{\bigl(\begin{smallmatrix}\ell & 1 & \ell{+}1 \\ 2 & 0 & -2\end{smallmatrix}\bigr)}
       {\bigl(\begin{smallmatrix}\ell & 1 & \ell{+}1 \\ 0 & 0 & 0\end{smallmatrix}\bigr)}.
\end{equation}
This is verified to machine precision ($< 10^{-14}$) for $\ell = 2$--$60$.
The ratio approaches unity for large $\ell$: it is $0.745$ at $\ell = 2$, $0.983$ at $\ell = 10$, and $0.999$ at $\ell = 50$.
\end{proposition}

The exact Gaunt formula for the $m$-summed squared coupling integral is
\begin{equation}\label{eq:gaunt}
  G(\ell) = \sum_m \left|\int Y_{\ell m}^* Y_{10} Y_{\ell+1,m}\,d\Omega\right|^2 = \frac{\ell+1}{4\pi}.
\end{equation}

\subsection{Contorsion Decoupling from Null Geodesics}\label{sec:contorsion}

\begin{proposition}[Contorsion--photon decoupling]\label{prop:contorsion}
For purely axial torsion $T^\mu{}_{\nu\rho} = \tfrac{1}{3}\varepsilon^\mu{}_{\nu\rho\sigma}\mathcal{A}^\sigma$, the contorsion coupling to null geodesics vanishes identically:
\begin{equation}\label{eq:contorsion-vanish}
  K^\mu{}_{\nu\rho}\,p^\nu p^\rho = \tfrac{2}{3}\,\varepsilon^\mu{}_{\nu\rho\sigma}\,\mathcal{A}^\sigma\,p^\nu p^\rho = 0.
\end{equation}
\end{proposition}

\begin{proof}
The Levi-Civita symbol $\varepsilon^\mu{}_{\nu\rho\sigma}$ is totally antisymmetric in $(\nu,\rho,\sigma)$, hence antisymmetric in $(\nu,\rho)$.  The photon momentum product $p^\nu p^\rho$ is symmetric in $(\nu,\rho)$.  Contraction of an antisymmetric tensor with a symmetric one vanishes identically.
\end{proof}

This result eliminates the direct contorsion coupling to CMB temperature anisotropies.  There is no ``tensor (contorsion $K\cdot\partial h$)'' contribution to the BiPoSH shape function.  The leading temperature modulation from a dipolar torsion gradient comes instead from the integrated Sachs--Wolfe (ISW) effect: the torsion stress-energy modifies the gravitational potential~$\Phi$, and the time-varying~$\dot\Phi$ generates a dipolar ISW signal.

\subsection{Shape Functions}\label{sec:shape-functions}

The physical shape function $r(\ell) \equiv \partial\!\ln C_\ell / \partial\!\ln\mathcal{A}_0$ is computed by the \emph{separate-universe} method: a superhorizon torsion gradient $\delta\mathcal{A}/\mathcal{A} = w_0\cos\theta$ causes each direction~$\hat{n}$ to see a slightly different cosmology, so we evaluate $C_\ell(\mathcal{A}_0 \pm \delta\mathcal{A})$ using a mini-Boltzmann solver (Sachs--Wolfe + Doppler + ISW line-of-sight integration with Eisenstein--Hu transfer function) and finite-difference.  The contorsion vanishing theorem (Proposition~\ref{prop:contorsion}) eliminates any direct geodesic coupling; the entire TT response comes through three physical channels operating simultaneously:

\begin{enumerate}[label=(\roman*)]
\item \emph{ISW channel} (dominates at $\ell \lesssim 10$): increasing $\Omega_A$ deepens the late-time potential decay, boosting $C_\ell$ at low~$\ell$.
\item \emph{Distance channel}: increasing $\Omega_A$ decreases the comoving distance $D_*$, compressing angular scales.  For $m_A/H_0 = 2.7$: $\partial\!\ln D_*/\partial\!\ln\mathcal{A}_0 = -0.27$.
\item \emph{Growth channel}: changing $\Omega_A$ modifies the growth function $D_+(a)$, affecting the potential at recombination and peak heights.
\end{enumerate}

\begin{center}
\begin{tabular}{rrrll}
\toprule
$\ell$ & $r_{TT}(\ell)$ & $r_{EB}(\ell)$ & $r_{EB}/r_{TT}$ & Dominant mechanism \\
\midrule
2  & $+2.19$ & $1$ & $0.46$ & ISW + distance \\
5  & $+1.41$ & $1$ & $0.71$ & ISW \\
10 & $+0.66$ & $1$ & $1.52$ & ISW (declining) \\
15 & $+0.24$ & $1$ & $4.3$  & ISW $\approx$ distance \\
20 & $-0.03$ & $1$ & ---    & zero crossing \\
30 & $-0.34$ & $1$ & ---    & distance (negative) \\
\bottomrule
\end{tabular}
\end{center}

\noindent The TT shape function is positive and steep at low~$\ell$ (power-law $r_{TT} \propto \ell^{-1.3}$ for $\ell \lesssim 15$), crosses zero near $\ell \sim 20$, and turns negative at higher~$\ell$ where the distance compression dominates.  All values are for $m_A/H_0 = 2.7$ (best-fit torsion mass); other masses give qualitatively similar shapes with the zero crossing shifting to higher~$\ell$ for larger $m_A$.  The EB shape function is exactly $r_{EB} = 1$ for all~$\ell$, since the Chern--Simons rotation angle is independent of angular scale.

\subsection{Five Pre-Registered Predictions}\label{sec:five-predictions}

\begin{description}
\item[Prediction 1: Birefringence dipole direction.]
The TT asymmetry direction is measured at $(l,b) \approx (225^\circ, -27^\circ)$.
The dipolar component of cosmic birefringence, when measured, must point in the same direction.

\item[Prediction 2: Dipolar birefringence amplitude.]
\begin{equation}
  \frac{\delta\alpha}{\alpha_0} = A_0^* = (2-\sqrt{3})^2 \approx 0.072.
\end{equation}
If $\alpha_0 \approx 0.3^\circ$, then $\delta\alpha \approx 0.02^\circ$.  This is not a free parameter.

\item[Prediction 3: EB BiPoSH parity alternation.]
\begin{equation}
  A^{EB}_{10}(\ell, \ell+1) \propto (-1)^{\ell+1}.
\end{equation}
A scalar modulation field produces no such alternation.

\item[Prediction 4: TT/EB shape split.]
The ratio $r_{EB}/r_{TT}$ grows from $\sim\!0.5$ at $\ell = 2$ through unity at $\ell \approx 10$ and diverges near the TT zero crossing at $\ell \sim 20$:
\begin{equation}
  r_{EB}(\ell)/r_{TT}(\ell) \approx 0.46,\; 0.71,\; 1.5,\; 4.3 \quad\text{at } \ell = 2,\,5,\,10,\,15.
\end{equation}
For $\ell > 20$, $r_{TT}$ changes sign while $r_{EB} = 1$ remains positive, so the EB and TT modulations are \emph{anti-correlated} at high~$\ell$---a distinctive signature of the two-channel mechanism that no single-field scalar modulation can produce.

\item[Prediction 5: TB/EB ratio.]
\begin{equation}
  \frac{A^{TB}_{10}(\ell,\ell+1)}{A^{EB}_{10}(\ell,\ell+1)} = \frac{C^{TE}_\ell}{C^{EE}_\ell}.
\end{equation}
This is independently measurable, providing a zero-parameter consistency check.
\end{description}

Predictions 1 and 2 are testable with LiteBIRD (launch 2032, data $\sim$2035).
Prediction 3 requires EB BiPoSH detection in individual $\ell$-bins (CMB-S4 combined with LiteBIRD).
Prediction 4 requires both TT and EB shape functions.
Prediction 5 functions as an internal consistency check.



% ============================================================================
% SECTION 13: SHAPE FUNCTION TENSION
% ============================================================================
\section{Shape Function Constraints and Tension with Planck Data}\label{sec:tension}

\subsection{The Observed Signal}

The Planck analysis fits a dipolar modulation $\Delta T(\hat{n})/T = A_0\,\hat{d}\cdot\hat{n}$ assuming $\ell$-independent amplitude.
The fitted amplitude decreases with $\ell_{\max}$:

\begin{center}
\begin{tabular}{rcc}
\toprule
$\ell_{\max}$ & $A(\ell_{\max})$ & $A/A(64)$ \\
\midrule
64  & 0.066 & 1.000 \\
128 & 0.054 & 0.818 \\
256 & 0.040 & 0.606 \\
512 & 0.024 & 0.364 \\
1024 & 0.012 & 0.182 \\
\bottomrule
\end{tabular}
\end{center}

\subsection{Comparison with the Boltzmann Shape Function}

The separate-universe Boltzmann computation (Section~\ref{sec:shape-functions}) gives a shape function $r_{TT}(\ell)$ that starts positive ($r_{TT}(2) \approx 2.2$), decays as $\sim\!\ell^{-1.3}$, crosses zero near $\ell \sim 20$, and turns negative at higher~$\ell$.  The $\ell_{\max}$-averaged prediction is:

\begin{center}
\begin{tabular}{rccc}
\toprule
$\ell_{\max}$ & $\hat{A}_{\mathrm{pred}}$ & $A_{\mathrm{obs}}$ & pred/obs \\
\midrule
5  & 0.12 & --- & --- \\
10 & 0.08 & --- & --- \\
20 & 0.03 & --- & --- \\
64 & $-0.03$ & 0.066 & $-0.5$ \\
\bottomrule
\end{tabular}
\end{center}

\noindent At low~$\ell$ ($\ell_{\max} \lesssim 10$), the torsion framework predicts the \emph{correct order of magnitude} ($A \sim 0.07$--$0.12$) and the observed \emph{declining trend} of $A(\ell_{\max})$ is qualitatively reproduced by the zero crossing.  However, the prediction turns negative at $\ell_{\max} \gtrsim 30$, while the data remain positive through $\ell_{\max} = 1024$.  This discrepancy could arise from limitations of the simplified Boltzmann code (Eisenstein--Hu transfer function, no Silk damping evolution, no polarization feedback) or from genuine tension with the data.

\subsection{Interpretation}

Three interpretations remain possible:

\begin{enumerate}[label=(\roman*)]
\item \emph{Partial match.}  The torsion framework successfully predicts $A(\ell_{\max} \lesssim 10) \sim 0.08$ and the declining $A(\ell_{\max})$ trend.  The zero-crossing prediction at $\ell \sim 20$ is sensitive to the transfer function approximation; a full CLASS/CAMB computation with torsion perturbations would determine whether it persists, shifts to higher~$\ell$, or softens.

\item \emph{Statistical contribution.}  The $3\sigma$ detection at $\ell_{\max} = 64$ includes a look-elsewhere effect.  The torsion contributes $A \sim 0.08$ at $\ell \lesssim 10$, and the observed persistence to $\ell = 64$ may reflect a statistical fluctuation on top of the torsion signal.

\item \emph{Different origin at high~$\ell$.}  The low-$\ell$ signal ($\ell \lesssim 15$) is from torsion; the high-$\ell$ continuation has a separate origin.  The birefringence predictions remain the primary observational test.
\end{enumerate}

We regard the match at low~$\ell$ as encouraging but the high-$\ell$ discrepancy as an open problem requiring full Boltzmann code refinement.

\subsection{The Shape Function as a Constraint}

The separate-universe computation demonstrates that a superhorizon dark-energy gradient affects $C_\ell$ through \emph{three} distinct channels (ISW, distance, and growth), not just the ISW effect assumed in earlier analyses.  The resulting $r_{TT}(\ell)$ with its zero crossing is qualitatively different from a pure power-law shape and provides a template for future CMB-S4 BiPoSH analyses.  Combined with the exact EB prediction $r_{EB} = 1$ (Chern--Simons), the framework offers a two-channel signature that is distinctive of the axial-torsion mechanism.



% ============================================================================
% SECTION 14: QUADRUPOLE--OCTUPOLE
% ============================================================================
\section{Quadrupole--Octupole Alignment}\label{sec:quadoct}

The preferred axes of the CMB quadrupole ($\ell = 2$) and octupole ($\ell = 3$) are aligned to within $\sim 2^\circ$ in the Planck data.
Because the dipolar torsion modulation couples $\ell$ to $\ell + 1$, it is natural to ask whether the $\ell = 2 \leftrightarrow \ell = 3$ coupling can produce the observed alignment.
We compute this exactly and find that it cannot.

The predicted cross-correlation coefficient is $\rho_{23} = 0.021$, shifting the alignment statistic by $\Delta S = 2.9 \times 10^{-4}$ ($0.002\sigma$ of the null distribution).
Reproducing the observed alignment would require $\rho_{23} \gtrsim 0.5$, corresponding to order-unity modulation of the CMB.

A further obstacle is directional: the hemispherical asymmetry dipole points toward $(l,b) \approx (225^\circ, -27^\circ)$, while the quadrupole--octupole axis points toward $(l,b) \approx (240^\circ, 63^\circ)$---a separation of $89^\circ$.
The two anomalies are associated with different directions on the sky.

The near-$m$-independence of the Gaunt coupling ($G_{\pm 2}/G_0 = 0.745$) means that the dipolar modulation does not preferentially enhance the zonal ($m = 0$) component that would create axial alignment.
The torsion framework does not explain the quadrupole--octupole alignment: the effect is too weak by a factor of $\sim 24$, in the wrong direction by $89^\circ$, and structurally incapable of the required mode coupling.



% ============================================================================
% SECTION 15: PRODUCT GEOMETRY AND GRADING VIOLATION
% ============================================================================
\section{Product Geometry and Grading Violation}\label{sec:product}

The full Dirac operator on the product geometry $K_4 \times F$ is
\begin{equation}\label{eq:Dfull}
  D_{\mathrm{full}}(t) = D_{\mathrm{space}}(t) \otimes \mathbf{1}_{N_F} + \gamma \otimes D_F,
\end{equation}
where $D_{\mathrm{space}}(t) = (1-t)D_{\mathrm{seq}} + t\,D_{\mathrm{hub}}$ interpolates from the Hamiltonian cycle ($t = 0$, where the grading $\gamma$ exists) to the hub-spoke ($t = 1$), and $D_F$ is the finite Dirac operator encoding Yukawa couplings.

The grading $\gamma = \diag(+1,-1,+1,-1)$ anticommutes with $D_{\mathrm{seq}}$ (bipartite) but not with $D_{\mathrm{hub}}$ (non-bipartite).
The anticommutator $C_{\mathrm{grad}} = \{D_{\mathrm{hub}}, \gamma\}$ is a rank-2 Hermitian matrix with eigenvalues $\{-2, 0, 0, +2\}$, coupling only the same-chirality vertices.

Squaring:
\begin{equation}
  D_{\mathrm{full}}(t)^2 = D_{\mathrm{space}}(t)^2 \otimes \mathbf{1} + \mathbf{1} \otimes D_F^2 + t \cdot C_{\mathrm{grad}} \otimes D_F.
\end{equation}
The cross-term $t \cdot C_{\mathrm{grad}} \otimes D_F$ is the unique coupling between spacetime geometry and internal particle physics, vanishing exactly at the bipartite point $t = 0$.

The Seeley--DeWitt coefficients $a_k(t) = \Tr(D_{\mathrm{full}}(t)^k)$ are all maximized at $t = 0$ and minimized at intermediate $t$: the cross-term drives the geometry toward lower total curvature.
Whether the symmetric vacuum ($t = 0$) is destabilized depends on the ratio $f_2/f_0$ of spectral action moments and the Yukawa coupling strength, with larger Yukawas favoring destabilization.



% ============================================================================
% SECTION 16: DISCUSSION
% ============================================================================
\section{Discussion}\label{sec:discussion}

\subsection{What Is Genuinely New}

The Pfaffian mechanism is a novel algebraic observation: the sign structure of perfect matchings in $K_4$ preferentially selects $(1,3)$ signature through the Hessian of a spectral-action-type functional.
This is a clean mathematical result independent of physics assumptions.
The 13/15 counting is exact, and the Pfaffian classification into Lorentzian ($\Pf \neq 0$) and non-Lorentzian ($\Pf = 0$) subgraphs is airtight.

The no-global-polarization theorem (Theorem~\ref{thm:noglobal}) shows that the contextuality of geometry is not imposed but derived: no single complex structure is compatible with all Lorentzian subgraphs.
The identification $H_\sigma = g_\sigma$ (Hamiltonian = metric) gives operational content to the emergence of time from local polarization.

The role of $\Lambda$ as a distinguishability regulator (Section~\ref{sec:lambda}) provides a new interpretation of the spectral cutoff: it does not define the geometry but sets the resolution scale at which the Hessian metric resolves the intrinsic eigenvalue gap $R = 7 + 4\sqrt{3}$.
The identification of the spectral action as a partition function whose Hessian is the Fisher information metric (Section~\ref{sec:fisher}) answers \emph{why this functional}: by \v{C}encov's theorem, the Fisher metric is the unique canonical metric on a statistical manifold, and the spectral action is the unique functional that produces it.  The Lorentzian signature flip is then the same mechanism as in Ruppeiner thermodynamic geometry, but driven by the Pfaffian rather than assumed via negative heat capacity.
The phase diagram (Section~\ref{sec:phase}) is a Morse-theoretic cascade (Section~\ref{sec:morse}): three bifurcation points at $s_1, s_2, s_3$ successively increase the Morse index from~0 (pre-geometric minimum) to~3 (Lorentzian saddle), with each bifurcation resolving one direction of the emergent lightcone.

The volume--curvature decomposition $a_2 = \frac{1}{4}[\Tr D^2]^2 - 2\,\Pf^2$ (Theorem~\ref{thm:a2decomp}) establishes the Pfaffian as a quadruple-role object: it simultaneously selects Lorentzian signature, carries the holonomy content of $a_2$, determines the discrete Ricci tensor via $\partial\Pf^2/\partial w$, and picks the unique Einstein--Lorentzian subgraph (Theorem~\ref{thm:coexist}).  The Einstein equation $\partial a_2/\partial w = \lambda w$ reduces to the condition that $\partial\Pf^2/\partial w \propto w$---a transparent geometric requirement that Pfaffian curvature be isotropic.

The Einstein--Lorentzian coexistence theorem (Theorem~\ref{thm:coexist}) is a new structural result: the sequential Hamiltonian cycle is the \emph{only} four-edge subgraph that simultaneously satisfies both the discrete Einstein equation and the Lorentzian condition $\Pf \neq 0$.  This persists at the full spectral action level: constrained critical points on hub-spokes are numerically driven to $\Pf = 0$ at every scale tested (Section~\ref{sec:fullaction}).  The matching class structure of $D^2$ (Section~\ref{sec:matching}) reveals the microscopic mechanism: the orientation of $D$ renders the Pf-negative matching invisible to $D^2$, breaking the $S_4$ vertex symmetry and explaining why $a_2$ distinguishes edge types.

The gap from weight-space geometry to spacetime geometry is now closed at the algebraic level and largely closed at the analytic level (Remark~\ref{rem:P4status}).  The walk decomposition of $\Tr(D^4)$ (Section~\ref{sec:walks}) separates volume (backtracking) from curvature (4-cycle holonomy), with the 4-cycle sign determined by the Pfaffian.  The vertex curvatures $\kappa(v) = \frac{1}{2}(D^4)_{vv}$ are local and converge to $\int R\sqrt{g}$ on refined triangulations via the Cheeger--M\"uller--Schrader theorem (Theorem~\ref{thm:convergence}).  What remains is the normalization constant matching $C_d$ to the Seeley--DeWitt coefficient $1/6$.

The $d = 4$ uniqueness theorem (Theorem~\ref{thm:d4unique}) shows that the entire Pfaffian signature mechanism is specific to four dimensions.  Newton's identity~\eqref{eq:a2-newton} decomposes $a_2$ into $\binom{N}{4}$ sub-Pfaffian$^2$ contributions: for $N = 4$ there is exactly one (the full Pfaffian), while for $N \geq 5$ the multiple sub-Pfaffians are generically sign-frustrated (Proposition~\ref{prop:frustration})---verified by explicit construction for $K_6$ and by $10^5$ Monte Carlo trials.  For odd~$N$, the Pfaffian does not even exist.  This converts the question ``why $d = 4$?'' from an assumption into a theorem.

\subsection{Summary of Killed Claims}

Several claims from earlier versions of this work have been corrected or removed based on rigorous verification:

\begin{enumerate}[label=(\roman*)]
\item \emph{Quadratic rigidity} ($d_n = n^2$ derived from first principles): downgraded to a well-motivated convention.  The Lorentzian selection mechanism is $\alpha$-independent.

\item \emph{Hemispherical asymmetry from torsion}: the contorsion coupling vanishes (Proposition~\ref{prop:contorsion}); the full separate-universe Boltzmann shape function $r_{TT}(\ell)$ matches the observed amplitude at $\ell \lesssim 10$ but develops a zero crossing near $\ell \sim 20$ that creates tension at $\ell_{\max} = 64$.  Refinement with a full Boltzmann code (CLASS/CAMB with torsion) is needed.

\item \emph{Quadrupole--octupole alignment}: the coupling is too weak by a factor of 24, in the wrong direction by $89^\circ$, and structurally incapable of the required alignment.

\item \emph{Modular theory identification}: downgraded from identification to structural parallel.

\item \emph{Hubbard model connection}: removed.  The formal algebraic analogy is interesting but the quantitative claims were unverified.

\item \emph{DHR superselection sectors}: retained as a corollary of $\Comm(\{D_\sigma\}) = \R$, not an independent theorem.
\end{enumerate}

\subsection{What Remains Strong}

The following results are fully verified and form the core of the paper:

\begin{center}
\begin{tabular}{lc}
\toprule
Result & Status \\
\midrule
13/15 Lorentzian genericity (Theorem~\ref{thm:main}) & verified \\
Pfaffian mechanism (Theorem~\ref{thm:pfaffian}) & verified \\
$D^2$ eigenvalues, ratio $R = 7+4\sqrt{3}$ & verified \\
Commutator formulas, $\Tr(C) = 0$ & verified \\
Continuous limit $a(x) = -4x^3$, $b(x) = -2x^3$ & verified \\
Positive mode scaling $n_+/\sqrt{N} \to 0.65$ & verified \\
Phase diagram and signature transitions & verified \\
Morse bifurcation cascade $\mu = 0 \to 1 \to 2 \to 3$ (Section~\ref{sec:morse}) & verified \\
$\Lambda$ as distinguishability regulator (Section~\ref{sec:lambda}) & verified \\
Fisher metric $=$ Hessian of partition function (Section~\ref{sec:fisher}) & identified \\
$a_2$ decomposition: $\Tr(D^4) = \tfrac{1}{2}[\Tr D^2]^2 - 4\Pf^2$ (Theorem~\ref{thm:a2decomp}) & proved \\
Discrete Ricci / Einstein equation (Proposition~\ref{prop:einstein}) & proved \\
Einstein--Lorentzian coexistence (Theorem~\ref{thm:coexist}) & proved \\
Full spectral action critical point degeneracy (Section~\ref{sec:fullaction}) & verified \\
Matching class structure in $D^2$ (Section~\ref{sec:matching}) & proved \\
Walk decomposition of $\Tr(D^4)$ (Section~\ref{sec:walks}) & proved \\
Curvature convergence on triangulations (Theorem~\ref{thm:convergence}) & proved \\
No global polarization (Theorem~\ref{thm:noglobal}) & proved \\
Maximal temporal contextuality (0/78 commuting) & verified \\
Cosmic birefringence $\alpha_0^{\max} = 0.41^\circ$ & verified \\
Amplitude $A_0 = (2-\sqrt{3})^2 \approx 0.072$ & verified \\
$m$-sum cancellation theorem & verified \\
Contorsion--photon decoupling (Proposition~\ref{prop:contorsion}) & proved \\
$d=4$ uniqueness via sub-Pfaffian frustration (Theorem~\ref{thm:d4unique}) & proved \\
Krein spectral triple $(\eta, \gamma, \mathcal{J})$, KO-dim $= 2$ (Section~\ref{sec:krein}) & constructed \\
Separate-universe $r_{TT}(\ell)$: $+2.2$ at $\ell=2$, zero at $\ell\sim20$, $r_{EB}=1$ & computed \\
Five falsifiable predictions & stated \\
\bottomrule
\end{tabular}
\end{center}

\subsection{Open Problems}

\begin{enumerate}[label=\textbf{P\arabic*}.]
\item \emph{Tensor harmonic transfer function} (resolved): The contorsion vanishing theorem (Proposition~\ref{prop:contorsion}) shows $K^\mu{}_{\nu\rho}p^\nu p^\rho = 0$, eliminating the direct coupling.  A separate-universe Boltzmann computation gives $r_{TT}(\ell) \approx 2.2\,\ell^{-1.3}$ for $\ell \lesssim 15$, crossing zero near $\ell \sim 20$ and turning negative at higher~$\ell$, driven by the interplay of ISW, distance, and growth effects.  The EB shape is exactly $r_{EB} = 1$.  Full CLASS/CAMB integration would refine the zero-crossing location.

\item \emph{Krein spectral triple completion} (resolved): The fundamental symmetry $\eta_\sigma$, chirality $\gamma_\sigma$, and real structure $\mathcal{J}_\sigma$ are constructed explicitly for all 13 Lorentzian subgraphs using the Schur decomposition of $D_\sigma$ (Propositions~\ref{prop:krein-eta}--\ref{prop:real-structure}).  All axioms verified; KO-dimension $= 2 \bmod 8$ uniformly.  The triple is inherently contextual (Theorem~\ref{thm:noglobal-krein}).

\item \emph{Discrete Einstein equations}: The $a_2$-level Einstein equation is now fully resolved on $K_4$: the sequential HC uniquely satisfies both $\partial a_2/\partial w = \lambda w$ and $\Pf \neq 0$ (Theorem~\ref{thm:coexist}).  The full (non-perturbative) spectral action exhibits the same Einstein--Lorentzian tension: constrained critical points on hub-spokes are driven to $\Pf = 0$ configurations at every scale tested.  The remaining problem is whether this tension relaxes on larger graphs $K_N$ where the subgraph topology is richer, or whether it constitutes a selection principle favouring edge-transitive (maximally symmetric) discrete geometries.

\item \emph{Continuum limit of $a_2$} (resolved): The walk decomposition (Section~\ref{sec:walks}) establishes $a_2 = \sum_v \kappa(v)$ as a sum of local vertex curvatures, with the 4-cycle contribution carrying the holonomy content.  Convergence $\sum_v \kappa_h(v) \to C_d \int R\sqrt{g}\,d^dx$ on refined triangulations follows from the Cheeger--M\"uller--Schrader theorem~\cite{CheegerMullerSchrader1984} (Theorem~\ref{thm:convergence}).  On $K_4$ itself, $a_2(K_4)/4$ is the fiber curvature density---no further limit is needed.  The remaining refinement is matching the normalization constant $C_d$ to the Seeley--DeWitt coefficient $1/6$ (Remark~\ref{rem:P4status}).

\item \emph{Higher-dimensional analysis} (resolved): Theorem~\ref{thm:d4unique} and Proposition~\ref{prop:frustration} show that the Pfaffian-based signature selection mechanism operates exclusively on $K_4$.  For $N < 4$, $a_2$ has no Pfaffian content; for $N \geq 5$, the $\binom{N}{4} \geq 5$ sub-Pfaffians in $a_2$ are generically sign-frustrated, and the full $\Pf(D_N)$ enters only at $a_{N/2}$, decoupled from the Einstein--Hilbert level.  The emergence of $d = 4$ is a consequence, not an assumption.
\end{enumerate}



% ============================================================================
% ACKNOWLEDGMENTS AND REFERENCES
% ============================================================================
\section*{Acknowledgments}

[To be added.]


\begin{thebibliography}{99}

\bibitem{Connes1994}
A.~Connes, \emph{Noncommutative Geometry}, Academic Press, 1994.

\bibitem{ChamseddineConnes1997}
A.H.~Chamseddine, A.~Connes,
``The spectral action principle,''
\emph{Commun.\ Math.\ Phys.}\ \textbf{186} (1997) 731--750.
[arXiv:hep-th/9606001]

\bibitem{ConnesMarcolli2008}
A.~Connes, M.~Marcolli,
\emph{Noncommutative Geometry, Quantum Fields and Motives},
AMS Colloquium Publications, 2008.

\bibitem{vandenDungen2012}
K.~van den Dungen, W.D.~van Suijlekom,
``Particle physics from almost-commutative spacetimes,''
\emph{Rev.\ Math.\ Phys.}\ \textbf{24} (2012) 1230004.

\bibitem{Minami2020}
Y.~Minami, E.~Komatsu,
``New extraction of the cosmic birefringence from the Planck 2018 polarization data,''
\emph{Phys.\ Rev.\ Lett.}\ \textbf{125} (2020) 221301.
[arXiv:2011.11254]

\bibitem{Eskilt2022}
J.R.~Eskilt,
``Frequency-dependent constraints on cosmic birefringence from the LFI and HFI Planck Data Release 4,''
\emph{Astron.\ Astrophys.}\ \textbf{662} (2022) A10.

\bibitem{LiteBIRD2023}
LiteBIRD Collaboration,
``Probing cosmic inflation with the LiteBIRD cosmic microwave background polarization survey,''
\emph{Prog.\ Theor.\ Exp.\ Phys.}\ \textbf{2023} (2023) 042F01.

\bibitem{Planck2015XVI}
Planck Collaboration,
``Planck 2015 results. XVI. Isotropy and statistics of the CMB,''
\emph{Astron.\ Astrophys.}\ \textbf{594} (2016) A16.
[arXiv:1506.07135]

\bibitem{Planck2019VII}
Planck Collaboration,
``Planck 2018 results. VII. Isotropy and statistics of the CMB,''
\emph{Astron.\ Astrophys.}\ \textbf{641} (2020) A7.
[arXiv:1906.02552]

\bibitem{Planck2018I}
Planck Collaboration,
``Planck 2018 results. I. Overview and the cosmological legacy of Planck,''
\emph{Astron.\ Astrophys.}\ \textbf{641} (2020) A1.

\bibitem{Erickcek2008}
A.L.~Erickcek, M.~Kamionkowski, S.M.~Carroll,
``A hemispherical power asymmetry from inflation,''
\emph{Phys.\ Rev.\ D} \textbf{78} (2008) 123520.

\bibitem{Schwarz2016}
D.J.~Schwarz, C.J.~Copi, D.~Huterer, G.D.~Starkman,
``CMB anomalies after Planck,''
\emph{Class.\ Quantum Grav.}\ \textbf{33} (2016) 184001.

\bibitem{Bennett2011}
C.L.~Bennett \emph{et al.},
``Seven-year Wilkinson Microwave Anisotropy Probe (WMAP) observations: are there cosmic microwave background anomalies?,''
\emph{Astrophys.\ J.\ Suppl.}\ \textbf{192} (2011) 17.

\bibitem{deOliveiraCosta2004}
A.~de Oliveira-Costa, M.~Tegmark, M.~Zaldarriaga, A.~Hamilton,
``Significance of the largest scale CMB fluctuations in WMAP,''
\emph{Phys.\ Rev.\ D} \textbf{69} (2004) 063516.

\bibitem{ConnesRovelli1994}
A.~Connes, C.~Rovelli,
``Von Neumann algebra automorphisms and time-thermodynamics relation in generally covariant quantum theories,''
\emph{Class.\ Quantum Grav.}\ \textbf{11} (1994) 2899--2917.

\bibitem{Rovelli2004}
C.~Rovelli, \emph{Quantum Gravity}, Cambridge University Press, 2004.

\bibitem{DHR1969}
S.~Doplicher, R.~Haag, J.E.~Roberts,
``Fields, observables and gauge transformations,''
\emph{Commun.\ Math.\ Phys.}\ \textbf{13} (1969) 1--23; \textbf{15} (1969) 173--200.

\bibitem{CheegerMullerSchrader1984}
J.~Cheeger, W.~M\"uller, R.~Schrader,
``On the curvature of piecewise flat spaces,''
\emph{Commun.\ Math.\ Phys.}\ \textbf{92} (1984) 405--454.

\bibitem{Ruppeiner1979}
G.~Ruppeiner,
``Thermodynamics: a Riemannian geometric model,''
\emph{Phys.\ Rev.\ A}\ \textbf{20} (1979) 1608--1613.

\bibitem{Cencov1982}
N.N.~\v{C}encov,
\emph{Statistical Decision Rules and Optimal Inference},
Translations of Mathematical Monographs \textbf{53}, AMS, 1982.

\end{thebibliography}


% ============================================================================
% APPENDICES
% ============================================================================
\appendix

\section{Code Listing: Commutator Construction}\label{app:code}

\begin{verbatim}
def build_commutator(N):
    """Build C = [A, A+] = AA+ - A+A from closed-form formulas.

    A = (I+S)Delta with Delta_nn = n^2, so:
      (AA+)_nn = n^4 + (n-1)^4  -> "output power"
      (A+A)_nn = 2n^4            -> "input power"

    C_nn = (AA+)_nn - (A+A)_nn = (n-1)^4 - n^4  for n < N
    C_NN = (AA+)_NN - (A+A)_NN = (N-1)^4
    C_{n,n+1} = -n^2(2n+1)
    """
    import numpy as np
    C = np.zeros((N, N))
    for n in range(1, N+1):
        idx = n - 1
        if n < N:
            C[idx, idx] = (n-1)**4 - n**4
        else:
            C[idx, idx] = (N-1)**4
        if n < N:
            C[idx, idx+1] = -n**2 * (2*n + 1)
            C[idx+1, idx] = -n**2 * (2*n + 1)
    return C
\end{verbatim}


\section{Notation}\label{app:notation}

\begin{center}
\begin{tabular}{lll}
\toprule
Symbol & Meaning & Section \\
\midrule
$\Delta$ & Diagonal growth matrix, $\Delta_{nn} = n^2$ & \ref{sec:canonical} \\
$S$ & Unilateral shift on $\R^N$ & \ref{sec:canonical} \\
$A = (I+S)\Delta$ & Canonical asymmetry operator & \ref{sec:canonical} \\
$C = [A, A^\dagger]$ & Commutator (standard ordering) & \ref{sec:commutator} \\
$D_\sigma \in \so(4)$ & Skew-symmetric edge matrix on 4-vertex subgraph & \ref{sec:pfaffian} \\
$I[w] = \Tr\exp(D^2/2)$ & Spectral action functional & \ref{sec:pfaffian} \\
$J_\sigma$ & Local complex structure (contextual polarization) & \ref{sec:noglobal} \\
$g_\sigma = D_\sigma J_\sigma$ & Metric (symmetric bilinear form) & \ref{sec:noglobal} \\
$H_\sigma = g_\sigma$ & Local Hamiltonian (= metric) & \ref{sec:noglobal} \\
$R = 7 + 4\sqrt{3}$ & Eigenvalue ratio of hub-spoke $D^2$ & \ref{sec:pfaffian} \\
$A_0^* = (2-\sqrt{3})^2$ & Inverse eigenvalue ratio / amplitude candidate & \ref{sec:birefringence} \\
$\alpha_T = 1/8$ & Torsion kinetic coefficient & \ref{sec:birefringence} \\
$\alpha_0$ & Isotropic cosmic birefringence angle & \ref{sec:birefringence} \\
$F(m_A)$ & Cosmological evolution factor & \ref{sec:birefringence} \\
\bottomrule
\end{tabular}
\end{center}


\end{document}
